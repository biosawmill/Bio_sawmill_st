\documentclass[a4paper,12pt]{article}
\usepackage[utf8x]{inputenc}
\usepackage[T1]{fontenc}
%\usepackage[T2A]{fontenc} % jei yra kirilica
\usepackage[hmargin={30mm,15mm},vmargin={20mm,20mm},bindingoffset=0mm]{geometry}
\usepackage[onehalfspacing]{setspace}
\usepackage[colorlinks=true, linkcolor=blue, citecolor=blue, urlcolor=blue, unicode]{hyperref}
%\parindent=7mm
\usepackage{graphicx}
\renewcommand{\refname}{Literatūros sąrašas} % article
%\renewcommand{\bibname}{Literatūros sąrašas} % report
\renewcommand{\contentsname}{Turinys}
\begin{document}
\thispagestyle{empty} % nerasomas psl. nr
\begin{center}
 VILNIAUS UNIVERSITETAS 
 
MATEMATIKOS IR INFORMATIKOS FAKULTETAS

MATEMATINĖS INFORMATIKOS KATEDRA

\vspace{4cm}

Projeckto vadovas \ \ \textbf{Vardas Pavardė} \\
\textbf{Vardas Pavardė} \\
\textbf{Vardas Pavardė} \\
\textbf{Vardas Pavardė} \\

\vspace{0.2cm}

Bioinformatikos studijų programos grupe BioSawmill



\vspace{3cm}
\textbf{\Large Dvimatė pjovimo optimizacija}\\
\textbf{\Large Projekto planas}

\vfill

Vilnius \ \  2015
\end{center}

\clearpage

\tableofcontents
\clearpage
%\maketitle 

\section*{Apžvalga}
\addcontentsline{toc}{section}{Apžvalga} % rasoma turinyje
\textbf{Aiste įpilk ką nors.}
\clearpage 

\section{ Dydžio ir pastangų vertinimas}
Atlikdami vertinamą naudosime rekomenduojama funkcinių taškų metodika. Kadangi kuriama "web" aplikacija, tai pirma naudosime atitinkamas "web" projekto sričių metrikas. Jų pavadinimai pateikiami angliškai:
\begin{enumerate}
\item \textbf{Number of static Web pages (NSW)}
\item \textbf{Number of dynamic Web pages (NDW)}
\item \textbf{Number of internal pages links (NIL)}
\item \textbf{Number of persistent data objects (NPDO}
\item \textbf{Number of external systems interfaced (NESI)}
\item \textbf{Number of static content objects (NSC)}
\item \textbf{Number of dynamic content objects (NDC)}
\item \textbf{Number of executable functions
(NEF)}
\end{enumerate}
Antra naudodami "web" metrika užpildysime funkciniams taškams skaičiuoti skirtas metrikas.
Jas pateikiama angliškai:
\begin{enumerate}
\item \textbf{Number of external imputs}
\item \textbf{Number of external outputs}
\item \textbf{Number of external inquiries}
\item \textbf{Number of internal logical files}
\item \textbf{Number of external interface files}
\end{enumerate}
Trečia matuojamas projekto sudėtingumas siekiant tikslesnio dydžio vertinimo. Tai daroma atsakant į klausimus priskiriant kokybinius vertes nuo 0 - No influence iki 5 - Essential. Klausimai pateikiami angliškai.
\begin{enumerate}
\item \textbf{Does the system require reliable backup and recovery?}
\item \textbf{Are data communications required?}
\item \textbf{ Are there distributed processing functions?}
\item \textbf{Is performance critical?}
\item \textbf{Will the system run in a existing, heavily utilized operational environment?}
\item \textbf{Does the system require on-line data entry?}
\item \textbf{Does the on-line data entry require the input transaction to be built over multiple screens or operations?}
\item \textbf{Are the master files updated on-line?}
\item \textbf{Are the inputs, outputs, files or inquiries complex?}
\item \textbf{Is the internal processing complex?}
\item \textbf{Is the code designed to be reusable?}
\item \textbf{Are conversion and installation included in the design?}
\item \textbf{Is the system designed for multiple installations in different organizations?}
\item \textbf{Is the application designed to facilitate change and ease of use by the user?}
\end{enumerate}
Ketvirta apskaičiuojami funkciniai taškai pagal formulę
$$ FP = count\ total \times [0.65 + 0.01 \times \sum(F_i)] $$
$FP$ - tai projekto dydžio vertinimas funkcinias taškais.


\subsection{Projekto dydžio apskaičiavimas}
 Tekstas ...
 \subsection{Projekto pastangų vertinimas}
 Tekstas su formule $y=x^2$...
\clearpage 

 \section{ Tvarkaraštis}
Antrasis skyrius sudarytas iš trijų poskyrių
\subsection{B1 pavadinimas}
 Tekstas ...
 \subsection{B2 pavadinimas}
 Poskyris B2 turi du skirsnius
 \subsubsection{B21 pavadinimas}
  Tekstas....
  \subsubsection{B22 pavadinimas}
  Tekstas....
 \subsection{B3 pavadinimas}
 Tekstas su nauja formule $y=x^3$...
\clearpage 
 
 




\end{document}
