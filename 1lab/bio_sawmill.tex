\documentclass[a4paper,12pt]{article}
\usepackage[utf8x]{inputenc}
\usepackage[T1]{fontenc}
%\usepackage[T2A]{fontenc} % jei yra kirilica
\usepackage[hmargin={30mm,15mm},vmargin={20mm,20mm},bindingoffset=0mm]{geometry}
\usepackage[onehalfspacing]{setspace}
\usepackage[colorlinks=true, linkcolor=blue, citecolor=blue, urlcolor=blue, unicode]{hyperref}
%\parindent=7mm
\usepackage{graphicx}
\renewcommand{\refname}{Literatūros sąrašas} % article
%\renewcommand{\bibname}{Literatūros sąrašas} % report
\renewcommand{\contentsname}{Turinys}
\usepackage[T1]{fontenc} 

% Lukas paketai
\usepackage{eso-pic}
\usepackage{indentfirst}
\usepackage{setspace}
\usepackage{placeins}
\usepackage{booktabs}% http://ctan.org/pkg/booktabs
\usepackage{tabularx}% http://ctan.org/pkg/tabularx
\usepackage[parfill]{parskip}
\usepackage[unicode]{hyperref}
\usepackage{hyperref}
\usepackage{tocloft}
\usepackage{graphicx}
\newcommand\AtPageUpperRight[1]{\AtPageUpperLeft{%
   \makebox[\paperwidth][r]{#1}}}
\usepackage[dotinlabels]{titletoc}
\usepackage[capposition=top]{floatrow}
\hypersetup{
    colorlinks,
    citecolor=black,
    filecolor=black,
    linkcolor=black,
    urlcolor=black
}
\usepackage{secdot}




\begin{document}


\renewcommand{\cftdot}{.}	
\renewcommand{\cftsecleader}{\cftdotfill{\cftdotsep}}

\thispagestyle{empty} % nerasomas psl. nr


\begin{center}
 VILNIAUS UNIVERSITETAS 
 
MATEMATIKOS IR INFORMATIKOS FAKULTETAS

MATEMATINĖS INFORMATIKOS KATEDRA

\vspace{4cm}

Projeckto vadovė \ \ \textbf{Aistė Čiplytė} \\
\textbf{Lukas Tutkus} \\
\textbf{Julius Daukšas} \\
\textbf{Dominykas Smaliukas} \\
\textbf{Robert Stankevič} \\

\vspace{0.2cm}

Bioinformatikos studijų programos grupe BioSawmill



\vspace{3cm}
\textbf{\Large Dvimatė pjovimo optimizacija}\\
\textbf{\Large Projekto planas}

\vfill

Vilnius \ \  2015
\end{center}



\clearpage

\tableofcontents
\clearpage
%\maketitle 
%\section*{Apžvalga}
%\addcontentsline{toc}{section}{Apžvalga} % rasoma turinyje




%\section{\centerline{„Dvimatė pjovimo optimizacija“ \\ PROJEKTO VIZIJA.}}
\setlength\parindent{24pt}
\begin{center}

\section[„Dvimatė pjovimo optimizacija“ PROJEKTO VIZIJA.]{\texorpdfstring{„Dvimatė pjovimo optimizacija“ \\PROJEKTO VIZIJA.}}

\end{center}
\setlength\parindent{14pt}

%\AddToShipoutPictureBG*{%
%  \AtPageUpperRight{\raisebox{-\height}{\frame{\includegraphics[width=4cm]{/home/l_/Pictures/%BioSawmill.png}}}}}


%\floatfoot{Uzi fig\= ura}

    Atsižvelgus į iškeltus užsakovo bazinius reikalavimus siūlome sukurti programą (naudojant WEB modelį) skirtą užsakovui ar jo klientams gauti optimizuotą medienos pjovimo informaciją. \\ 

\vspace{-5mm}
    Siūlome turėti dvi šios sistemos  versijas „BETA“ (terminuotą laike, klientams išbandyti pagrindinius sistemos funkcionalumus) ir PILNĄJĄ (su visais įrankio funkcionalumais)\\
	
	\vspace{-5mm}
	Sistema administruojama jos užsakovo administratoriaus teisėmis (galimybė matyti sistema besinaudojančių klientų duomenis (e-mail, aktyvacijos pradžios/pabaigos laikas, teisių lygis, pastabos), kurti/redaguoti/trinti klientų duomenis. Užsakovo klientus siiūlome skirstyti į privačius asmenis ir įmones, kuriems pagal teises būtų sukuriama galimybė turėti prieigą prie sistemos apibrėžtam/nurodomam administratoriaus skaičiui vartotojų.\\
	
	\vspace{-5mm}
	BETA versijos vartotojas užsiregistravęs savo elektorniniu paštu prisijungia prie sistemos su labai ribotomis teisėmis (optimizuoto pjovimo skaičiavimas, vartotojo aktyvacijos galiojimo peržiūra, PILNOS versijos užsakymo langas).Optimizuoto pjovimo skaičiavimo lange jis turi galimybę nurodyti reikalingų detalių ilgius (mm), aukščius (mm) ir kiekius. Sistema optimizuos detalių išpjovimą iš standartinių panelių 1200x2500 ir 1200x3050, beiišvęs informaciją ekrane.  Vartotojui pasirinkus norimą variantą, sistema parodys ekrane pjovimo planą (spausdinimo galimybės nėra). 
\begin{center}
	PILNA versija priklausomai nuo teisių leidžia:
	\begin{itemize}
		\item Papildomų vartotojų administravimą (kūrimas/redagavimas/šalinimas)
		\item Optimizuoto pjovimo skaičiavimas (standartinis funkcionalumas)
		\item Papildomo standartinio panelių dydžių įvedimas/redagavimas/šalinimas
		\item Galimybė pasirinkti kurioms panelėms atlikti skaičiavimus
		\item Galimybė atsispausdinti/išsaugoti bendrinę informaciją/ataskaitą (Vartotojo  įvestas ilgis, plotis, kiekis, panelės/-ių dydis/-žiai ir optimalus ruošinių kiekis pirmoje panelėje)
		\item Istorijos peržiūra (išsaugotų skaičiavimų istorija su galimybe atsispausdinti planą)
		\item Pasirinkto plano atvaizdavimas naujame ekrane (PDF formatu)
		\item Galimybė atsispausdinti optimizuotus pasirinkto panelės dydžio planą.
	\end{itemize}
	Dėl sistemos aplinkos ji gali būti lengvai versijuojama ir tobulinama atsižvelgiant į tolimesnius užsakovo poreikius nepažeidžiant klientų istorijos .
\end{center}
	




\section{ Dydžio ir pastangų vertinimas}
Atlikdami vertinamą naudosime rekomenduojama funkcinių taškų metodika. Kadangi kuriama "web" aplikacija, tai pirma naudosime atitinkamas "web" projekto sričių metrikas, noredami nustatyti apytikslei aplikacijos sudetingumą. Jų pavadinimai pateikiami angliškai:
\begin{enumerate}
\item \textbf{Number of static Web pages (NSW)}
\item \textbf{Number of dynamic Web pages (NDW)}
\item \textbf{Number of internal pages links (NIL)}
\item \textbf{Number of persistent data objects (NPDO)}
\item \textbf{Number of external systems interfaced (NESI)}
\item \textbf{Number of static content objects (NSC)}
\item \textbf{Number of dynamic content objects (NDC)}
\item \textbf{Number of executable functions
(NEF)}
\end{enumerate}
Antra naudodami "web" metrika užpildysime funkciniams taškams skaičiuoti skirtas metrikas.
Jas pateikiama angliškai:
\begin{enumerate}
\item \textbf{Number of external imputs}
\item \textbf{Number of external outputs}
\item \textbf{Number of external inquiries}
\item \textbf{Number of internal logical files}
\item \textbf{Number of external interface files}
\end{enumerate}
Trečia matuojamas projekto sudėtingumas siekiant tikslesnio dydžio vertinimo. Tai daroma atsakant į klausimus priskiriant kokybinius vertes nuo 0 - No influence iki 5 - Essential. Klausimai pateikiami angliškai.
\begin{enumerate}
\item \textbf{Does the system require reliable backup and recovery?}
\item \textbf{Are data communications required?}
\item \textbf{ Are there distributed processing functions?}
\item \textbf{Is performance critical?}
\item \textbf{Will the system run in a existing, heavily utilized operational environment?}
\item \textbf{Does the system require on-line data entry?}
\item \textbf{Does the on-line data entry require the input transaction to be built over multiple screens or operations?}
\item \textbf{Are the master files updated on-line?}
\item \textbf{Are the inputs, outputs, files or inquiries complex?}
\item \textbf{Is the internal processing complex?}
\item \textbf{Is the code designed to be reusable?}
\item \textbf{Are conversion and installation included in the design?}
\item \textbf{Is the system designed for multiple installations in different organizations?}
\item \textbf{Is the application designed to facilitate change and ease of use by the user?}
\end{enumerate}
Ketvirta apskaičiuojami funkciniai taškai pagal formulę
$$ FP = count\ total \times [0.65 + 0.01 \times \sum(F_i)] $$
$FP$ - projekto dydžio vertinimas funkcinias taškais.


\subsection{Projekto dydžio ir pastangų apskaičiavimas}

NSW skaičiuojamas:
	$$NSW \approx 3 \times NDW$$
NDW skaičiuojamas:
	$$NDW \approx 5 $$
NIL skaičiuojamas:
	$$NIL \approx 4 + 2 $$
NPDO skaičiuojamas:
	$$NPDO \approx NDW\times 3$$
NESI skaičiuojamas:
	$$NESI \approx 1 $$
NSC skaičiuojamas:
	$$NSC \approx (NSW+NDW)\times 5$$
NDC skaičiuojamas:
	$$NDC \approx NDW\times 3$$
NEF skaičiuojamas:
	$$NEF \approx 0$$

Number of external inputs:\\
NDW\\
Number of external outputs:\\
NDC\\
Number of external inquiries:\\
NESI\\
Number of internal logical files:\\
NPDO\\
Number of external interface files:\\
1

\begin{center}
  \begin{tabular}{ l | l | l || r }
    \hline
    Domain & count & Weight & FP \\ \hline
    External inputs & 5 & 3 & 15 \\ \hline
    External outputs & 15 & 1 & 15 \\ \hline
    External inquiries & 1 & 10 & 10 \\ \hline
    Internal logical files & 15 & 1 & 15 \\ \hline
    External interface files & 1 & 7 & 7 \\ \hline
    \multicolumn{3}{ l }{Total:} & 62 \\
    \hline
  \end{tabular}
\end{center}

\begin{enumerate}
 \item $3$
 \item $1$
 \item $0$
 \item $1$
 \item $0$
 \item $5$
 \item $3$
 \item $3$
 \item $4$
 \item $5$
 \item $0$
 \item $0$
 \item $0$
 \item $3$
 \end{enumerate}
 Viso: $28$
 
 $$FP \ = \ 62 \ \times \ 0.93 \ =\  57.66 $$

EI:
$$ S_1 =( 4 + 4 \times 5 + 6)\div 6$$
EO:
$$ S_2 =( 12 + 4 \times 15 + 17) \div 6$$
EQ:
$$ S_3 =( 1 + 3 \times 1 + 2 \times 2)\div 6$$
ILF:
$$ S_4 =( 12 + 4 \times 15 + 17)\div 6$$
EIF:
$$ S_5 =( 1 + 4 \times 1 + 2)\div 6$$\\
 \begin{center}
  \begin{tabular}{ l | l | l | l | l | l || r }
    \hline
    Domain & Opt. & Likely & Pess. & Est. & Weight & FP \\ \hline
    External inputs & 4 & 5 & 6 & 5 & 3 & 15 \\ \hline
    External outputs & 12 & 15 & 17 & 15 & 1 & 15 \\ \hline
    External inquiries & 1 & 1 & 2 & 1.5 & 10 & 15 \\ \hline
    Internal logical files & 12 & 15 & 15 & 0 & 1 & 15 \\ \hline
    External interface files & 1 & 1 & 2 & 1 & 7 & 7 \\ \hline
    \multicolumn{6}{ l }{Total:} & 67 \\
    \hline
  \end{tabular}
\end{center}

	 $$FP \ = \ 67 \ \times \ 0.93 \ =\  62.31 $$
	 
	 \href{http://www.qsm.com/resources/performance-benchmark-tables#Business%20Function%20Point}{Function Points per Person Monht table}.

	$E \ = 62.31 \div 7.1 = 8.8$

\clearpage 



\section{ Tvarkaraštis}

\subsection{Tvarkaraščio užsiėmimai}

\begin{frame}
\centering
\hspace*{-3 cm}
\centering
\label{my-label}
\begin{tabular}{|l|l|l|}
\hline
Darbuotojas & Darbas & Valandų skaičius \\ \hline
Projekto valdytojas & Prižiūri darbus, komunikuoja su užsakovu
\end{tabular}
\end{frame}


\subsection{Projekto planas}

\begin{frame}
\centering
\hspace*{-3 cm}

\label{my-label}
\begin{tabular}{|l|l|}
\hline
Diena          & Planas                                                                                                                       \\ \hline
Pirmadienis    & \begin{tabular}[c]{@{}l@{}}Darbų tęsimas, bei sprint'o savaitės planavimas\\ užduočių visai savaitei planavimas\end{tabular} \\ \hline
Antradienis    & Peržiūrėjimas užduočių, jų atlikimas, siuntimas testuotojams                                                                 \\ \hline
Trečiadienis   & Peržiūrėjimas užduočių, jų atlikimas, siuntimas testuotojams                                                                 \\ \hline
Ketvirtadienis & Peržiūrėjimas užduočių, jų atlikimas, siuntimas testuotojams                                                                 \\ \hline
Penktadienis   & \begin{tabular}[c]{@{}l@{}}sprint' o savaitės darbų pristatymas, \\ atliktų neatliktų užduočių tobulinimas.\end{tabular}     \\ \hline
\end{tabular}
\end{frame}
\\


Projektas vykdomas Agile projektavimo tipu.
Kadangi jis leidžia greitai spręsti iškylusias problemas bei beveike be pastangų tobulinti projektą.

Projekto kaina apskaičiuojama taip: \\
Kaina(LTL) = Pastangos(mėnesiai žmogui) * Darbo mėnesio kaina(LTL/Mėnesį). \\\\
Įmonės mėnesio darbo kaina, 175000 LTL, tad projekto
 kaina būtų: \\
Kaina = 1,93 * 17500 = 33775 LTL  
 
 
\subsection{Kontroliniai taškai}
Projekto etapai savo ruožtu gali turėti keletą kont
rolinių taškų. Juose nurodomi suplanuoti terminai, 
iki kada turėtų būti atlikti kažkurie projekto darb
ai, ir suplanuotos biudžeto išlaidos. Kontroliniai 
taškai padeda įvertinti projekto stovį duotais laik
o momentais. 
\begin{itemize}
\item
Projekto planavimas ir kitų klausimų sprendimas - 1
 savaitė (spalio 1d. - spalio 8d.) Išlaidos 
5 tūkstančiai litų. 
\item
Reliacinės duomenų bazės kūrimas - 1 savaitė (spali
o 8d. – spalio 15d.) Išlaidos 4 
tūkstančiai litų. 
\item
Grafinės sąsajos ir serverio kūrimas – 1 savaitė (s
palio 15d. – spalio 22d.) Išlaidos 4 
tūkstančiai litų. 
\item
Kreipinių kūrimo ir valdymo programinio komponento 
kūrimas -  1 savaitė (spalio 22d. – 
spalio 29d.) Išlaidos 4 tūkstančiai litų. 
\item
Vartotojų valdymo programinio komponento kūrimas -1
 savaitė (spalio 29d. – lapkrčio 5d.) 
Išlaidos 4 tūkstančiai litų. 
\item
Statistinių duomenų programinio paketo kūrimas -  1
 savaitė (lapkričio 5d. – lapkričio 12d.) 
Išlaidos 4 tūkstančiai litų. 
\item
Sistemų integracija į vieną sistemą, perkėlimas į U
test aplinką ir testavimas - 2 savaitės 
(lapkričio 12d. - lapkričio 26d.) Išlaidos 10 tūkst
ančiai litų. 
	\item Sistemos pristatymas užsakovui ir paskutiniai testa
vimai - 2 savaitės (lapkričio 26d. - 
gruodžio 10d.) Išlaidos 7460 litų. 
\end{itemize}
 


\section{Konfiguracin\. e kontrol\. e}
Kekvieno produkto gyvavimo cikle atsiranda nei\v sviangiam\k u pasikeitim\k u.\\
Konfiguracin\. e kontrol\. e u\v zikrina, kad tie pasikeitimai b\= utu minimal\= us.
\v Siame skyriuje aptarsiu galimus pasikeitimai i\v s u\v zakovo pus\. es bei kaip juos kontroliuoti. \\
\\
U\v zsakovas gali nuspr\k esti
\begin{itemize}
	\item pakeisti produkto informacij\k a.
	\item kada ir kiek i\v spl\. esti, pagreitinti programos funkcionalum\k a.
	\item koki\k i servis\k u papildymo tr\= uksta.
\end{itemize}

Kekvin\k a u\v zsakovo sprendim\k a turime ivykdyti. Problema tame, kad reikia atsekti kada ir kur ivykdyti pakeitim\k a.\\
Problemos sprendimas ir b\= utu konfiguracin\. es kontrol\. es naudojimas.\\
Ji u\v ztikrina, kad informacija saugojama, kontroliuojama sistemos k\= urimo, keitimo ir testavimo metu ir taip pat ar atitinka vykdomus \v reikalavimus. \\
Taip pat u\v ztikrinti kad informacij\k a apie projekta laisvai pasiekiama.\\

Konfiguracin\k e kontrol\. es u\v zuduotys:
\begin{enumerate}
	\item Rasti kokios projekto dalys gali keistis.
	\item kontroliuoti vien\k a ar kelis rastus.
	\item lengvinti skirting\k u versij\k u aplikacij\k u konstrukcij\k a. 
	\item u\v ztikrinti kad projekto kokybe palaikoma, kai konfiguracija atnaujinama.
\end{enumerate}


Konfiguracijos kontrole u\v ztikrina programos konfiguracijos element\k u identifikavim\k a, pasikeitimo kontrol\k e, versijos kontrol\k e, konfiguracijos audit\k a ir ataskaita tiek projekto k\= urimo metu, tiek palaikymo metu.\\\\


\large\textbf{Projekte kontrol\. e u\v ztikrinama}
\normalsize
\begin{enumerate}
	\item Programos dokumentacijai.
	\item Programos analiz\. es failai.
	\item Programos diegimo failam.
	\item Programavimo kodą.
	\item Internetinės svetainės kodas.
\end{enumerate}

\clearpage

\section{Versij\k u kontrol\. e}

Versij\k u kontrolei naudojamas "Github".\\
Ne tik programos kodui versijonuoti bet taip pat ir apie tai informuoti bendradarbius.\\

\large\textbf{Joje laikomi failai:}
\normalsize
\begin{enumerate}

	\item Vadyba ( organizacijos strukt\= ura, architekt\= uros informacija ).
	\item Modeliavmas ( analiz\. e, dizainas ).
	\item Konstrukcijos ( \v saltinio kodas, kompiliavimo instrukcijos )
	\item Testai ( skriptai, rezultatai, kokyb?s matavimai ).
	\item Dokumentai (projekto, dizaino dokumentai, naudojimo \v zinynas ).
	\item Projekto valdymas ( projekto trukm\. e, tvarkara\v stis, auditas ).
		
\end{enumerate}
\clearpage 


\section{Pakeitimų valdymo procedūra}

\large\textbf{Pakeitimų skirstymas}
\begin{enumerate}
	\item funkcijos arba aplikacijos ar rezultat? atvaizdavimo patobulinimai ar klaid\k u juose pataisymas
	\item funkcijos ar atvaizdavimo patobulinimai, kurie turi ?takos kitoms projekto programoms
	\item funkcij\k u ir naudojam\k u duomen\k u pakeitimai, kurie stipriai pakei\v cia, patobulina ar supaprastina web aplikacij\k u ar rezultat\k u atvaizdavim\k a
	\item dideli dizaino ar navigacijos web aplikacijoje pakeitimai, kurie stipriai \k itakoti vartotoj\k a.
\end{enumerate}

Šioje pastraipoje yra aprašomas algoritmas, kaip bus veikiama, iškylus poreikiui atlikti pakeitimus. 
Susidūrus su noru kažką pakeisti, jam, pirma, bus priskirta apimties/svarbumo klasė. Nepriklausomai nuo priskirtos klasės, jeigu pakeitimui įgyvendinti nėra nepritariama, tai jis bus pradedamas vykdyti pažingsniui: 
\begin{enumerate}
	\item taisytinas objektas (programos kodas) bus išsikeliamas
 (check-out)
 	\item atliekami pakeitimai
 	\item ištestuojami pakeitimai
 	\item pakeitimai įkeliami į repozitoriją (check-in)
 	\item paleidžiama web aplikacija su atliktu pakeitimu.
\end{enumerate}
Priklausomai nuo priskirtos klasės, dar prieš išsikeliant taisytinus objektus, atliekamos tokios procedūros: 
\begin{itemize}
	\item 2-os kl. pakeitimui atilikti tikrinami visi objektai, kurie susiję su keistinu objektu.
		Įvertinama, kokią įtaką jiems tie pakeitimai turės. kiek tuomet pakeitimų (derinimų) reikės atlikti šiuose susijusiuose objektuose.		
	\item 3-os kl. pakeitimui atlikti rašomas glaudus pakeitimo aprašymas.\\
	Pateikiamas visiems projekto vykdytojams, kad šie susipažintų ir pritartų, ir nesant jų pritarimui, apie pakeitimą diskutuojama ieškant kompromiso arba jis visiškai atmetamas.
	\item 4-os kl. pakeitimui atlikti rašomas glaudus pakeitimo aprašymas.\\
	Pateikiamas visiems užsakovams, kad šie peržiūrėtų ir pritartų, o nesant pritarimui, klausomasi jų siūlymų, ieškoma abi puses patenkinančio susitarimo ir jam susiformavus, pakeitimas atliekamas. 
\end{itemize}
 
\section{Resursų planas}



Žmones esantys kompanijoje, darbuotojai.


\small
\begin{frame}
\centering
\hspace{-3cm}
\label{my-label}
\begin{tabular}{|c|c|l|c|}
\hline
\textbf{\begin{tabular}[c]{@{}c@{}}Darbuotojas/\\ Pareigos\end{tabular}} & \textbf{\begin{tabular}[c]{@{}c@{}}Darbuotojų\\ kiekis\end{tabular}} & \multicolumn{1}{c|}{\textbf{Darbo pobūdis}}                                                                                                                                                        & \textbf{\begin{tabular}[c]{@{}c@{}}Darbo\\ valandų\footnotemark[1]\end{tabular}} \\ \hline
Administratorius                                                         & 1                                                                    & \begin{tabular}[c]{@{}l@{}}Sprendžia iškilusias techninės ir programinės \\ įrangos problemas. Atsakingas už darbuotojams \\ reikalingų resursų pasiekiamumą.\end{tabular}                             & 52                                                               \\ \hline
Analitikas                                                               & 1                                                                    & \begin{tabular}[c]{@{}l@{}}Tikrina ar programuotojų sukurtas sistemos \\ funkcionalumas atitinka kliento lūkesčius. Iš dalies \\atlieka testuotojo vaidmenį.  Ruošia dokumentaciją.\end{tabular}    & 100                                                              \\ \hline
Dizaineris                                                               & 1                                                                    & \begin{tabular}[c]{@{}l@{}}Kuria pradinį svetainės dizainą, o projekto eigoje keičia \\ dizainą pagal kliento reikalavimus. Suteikia profesionalias \\ konsultacijas programuotojams.\end{tabular} & 44                                                               \\ \hline
Programuotojas-testuotojas                                               & 2                                                                    & Rašo programinį kodą, o vėliau jį testuoja.                                                                                                                                                        & 120x2                                                            \\ \hline
Programuotojas-testuotojas                                               & 1                                                                    & \begin{tabular}[c]{@{}l@{}}Rašo programinį kodą, o vėliau jį testuoja. Paskutiniame \\ etape bus atsakingas už produkto palaikymo užtikrinimą.\end{tabular}                                         & 140                                                              \\ \hline
Projekto vadovas                                                         & 1                                                                    & \begin{tabular}[c]{@{}l@{}}Pagrindinis asmuo bendraujantis su klientu. Atsakingas už\\ projekto vystymą, komandos narių parinkimą. Konsultuoja\\ komandos narius.\end{tabular}                     & 100                                                              \\ \hline
Sistemos inžinierius                                                     & 1                                                                    & \begin{tabular}[c]{@{}l@{}}Pagal pateiktus ir išanalizuotus kliento poreikius sudaro \\ sistemos architektūrą (įskaitant ir techninę dalį). \\ Konsultuoja programuotojus.\end{tabular}                & 100                                                              \\ \hline
\end{tabular}
\end{frame}
\vspace{1cm}
\\
\textbf{Pastaba}: laikoma, kad žmogus dirbantis nepilnu etatu, tuo pačiu metu dirba prie kitų įmonės projektų.
\footnotetext[1]{Darbo valandų skaičius apskaičiuotas atsižvelgus į darbuotojų numatytas darbo valandas kiekviename etape.}
\clearpage
Techninės įrangos poreikiai:
\begin{itemize}
\item Kekvienam darbuotojui suteikiamas asmeninis kompiuteris, kuriame sudiegta reikalinga programinė įranga.
\end{itemize}

	

\section{Rizikos valdymas}

\subsection{Identifikavimas}



\FloatBarrier
%\begin{figure}[H]

\begin{frame}
\centering
\hspace*{-3 cm}
%\caption{Rizik\k u valdymas}
\small
\setlength\tabcolsep{3pt}
\begin{tabular}{|l|l|l|l|l|l|}
		\hline
 		Nr. & Rizik\k u prie\v zastys & Rizikos & Kategorija & tikimyb\. e
 		& \k Itaka  \\ \hline
 		1. &
 		\begin{tabular}[c]{@{}l@{}}
			Kalbos barjeras, \\
			u\v zsakovo laiko tr\= ukumas
 		\end{tabular} &
 		\begin{tabular}[c]{@{}l@{}}
			Komunikavimo tr\= ukumas
 		\end{tabular} & 
 		PR 	 & 
 		didel\. e & 
 		1	\\ \hline
 
 		2. & 			
 		\begin{tabular}[c]{@{}l@{}}
			N\. era atsiskaitymo ta\v sk\k u,\\
			ma\v zai pasitarim\k u
 		\end{tabular}  & 
 		Produktas neatitinka reikalavim\k u &
 		VR	 & 
 		didel\. e &
 		2	\\ \hline
 		
 		3. &
 		\begin{tabular}[c]{@{}l@{}} 
 			Neu\v ztikrinamos geros \\darbo s\k alygos \\ 
 		\end{tabular} &
		\begin{tabular}[c]{@{}l@{}} 
 			Darbuotojai palieka darboviet\. e, \\ 
 			dažnas darbuotoj\k u keitimas, \\
 			produkto v\. elinimas \\
 			mai\v sytas produktas \\
 		\end{tabular} &
  		PR 	 & 
 		didel\. e& 
 		3	\\ \hline
 
 		4. &
 		\begin{tabular}[c]{@{}l@{}} 
 			Trumpos apklausos, ma\v zai\\
 			testavimo priimant\\
 			naujus darbuotojus 
 		\end{tabular} &
 		
 		\begin{tabular}[c]{@{}l@{}} 
			Nekvalifikuoti darbuotojai 			
  		\end{tabular} & 
 		DR 	 & 
 		vidutin\. e & 
 		2	\\ \hline
 		
 		5. &
 		\begin{tabular}[c]{@{}l@{}}
			Galimybi\k u pervertimas  \\
			Neai\v skus darbo planas.
 		\end{tabular} &
 		
 		\begin{tabular}[c]{@{}l@{}}	
 			Produkto kokyb\. e pablog\. eja \\
 			nei\v sbaigtos produkto dalys\\ 
 		\end{tabular} &
 		PR 	 & 
 		ma\v za& 
 		1 \\ \hline
 		
 		6. &
 		\begin{tabular}[c]{@{}l@{}} 
 			Darbas vir\v svaland\v ziais, \\
 			blogas darbu planas.
 		\end{tabular} &
		\begin{tabular}[c]{@{}l@{}} 
 			Klaid\k u \k ivelimas, \\
 			Laiko praleidimas
 		\end{tabular} &
  		PR 	 & 
 		vidutin\. e& 
 		3	\\ \hline
 			
 		7. &	 
 		Darbuotoj\k u nekomunikavimas & 
		\begin{tabular}[c]{@{}l@{}}
			Produkto kokybe blogesn\. e,\\
 			programinis nesuderinamumas \\
		\end{tabular}  & 
 		PR 	 & 
 		ma\v za & 
 		3	\\ \hline
 		
 		8. &
 		\begin{tabular}[c]{@{}l@{}}
			Atsiskaitymo data atkeliama \\
			anks\v ciau
 		\end{tabular}  & 
 		V\. elavimas &
 		VR	 & 
 		ma\v za &
 		3	\\ \hline
 		
 		9. &
 		\begin{tabular}[c]{@{}l@{}}
			Dokumento ra\v symo atidejimai.
 		\end{tabular}  & 
 		Produkto dokumentacija nepateikiama laiku &
 		VR	 & 
 		ma\v za&
 		3	\\ \hline
 
 		
	\end{tabular}%
\end{frame}
	\\\\ \k Itakos dydžiai: 
		1 - Katastrofinis,
		2 - Kritikalus,
		3 - Ne\v zymus.\\
		PR - Produkto Rizika. DR - Darbuotoj\k u Rizika. VR - Verslo rizika. \\\\
	
\subsection{Valdymo strategija}
Identifikave galimas rizik\k u prie\v zastis paruo\v siame strategija, kaip jas valdysim.
\begin{itemize}
  \item Rizikos suma\v zinimas.
  \item Rizikos steb\. ejimas.
  \item Rizikos valdymas, nenumatyt\k u atv\. ej\k u planavimas.
\end{itemize}

\newpage

\subsection{Rizikos suma\v zinimas, }
Stengiamasi suma\v zinti projekto valdymo galioje esan\v cias rizikas.

\begin{enumerate}

	\item Komunikavimo tr\= ukumas . \\
  		Svarbu kalb\. eti, klausti, susitarti d\. el vietos ir laiko susitikimam.\\
  		Sprendimas variantas b\= utu surasti \v zmog\k u, kuris gal\. etu sukomunikuoti \\
  		arba pasi\= ulyti \v kita komunikavimo b\= uda pvz.: komunikuoti internetu. 
  		
  	\item Produktas neatitinka reikalavim\k u . \\
  		Sudaryti kontrolinius ta\v skus, atsiskaitymus, kuriuose kekvienas \\ 
  		darbuotojas ar j\k u grup\. e parašo kiek nuveik\. e.\\
  		Daryti konferencijas, kuriose spren\v ziamos atsiradusios problemos.
  		
	\item Darbo vietos neu\v ztikrinimas.\\
  		U\v ztiktinti geras darbo salygas. Dažniau daryti apklausas apie darb\k a,\\ 
  		\k ivertinti iš naujo darbuotoj\k u atlyginimus.\\
  		Tam ivykus geriausia b\= utu pasamdyti atitinkam\k a darbuotoj\k a ir u\v ztikrinti,\\
  		kad nepasikartot\k u senojo darbuotojo i\v s\. ejimo prie\v zastis, jei \k imanoma.
  		
	\item Nekvalifikuoti darbuotojai. \\
  		Tinkama\k u apklaus\k u u\v ztikrinimas, irodan\v ci\k u \k idarbinamojo sugeb\. ejimus. \\ Per\v zvelgti projekto dokumentacija i\v staisyti netikslumus.\\
  		Gavus nekvalifikuot\k a darbuotoj\k a, svarbu suteikti reikaling\k a asmen\k i, \\
  		kuris suteiktu jam reikalinga informacija ir pri\v ziur\. etu darbus.

	\item Prasta produkto kokybe, nei\v sbaigtos produkto dalys.\\
		\k ivertinti projekto sud\. etingum\k a, nustatyti ar atitinka galimybes.\\
		Per v\. elai  \k ivertinus b\= utina prane\v sti u\v zsakovui ir ie\v skoti bendro 				sprendimo.
  		
	\item Klaid\k u \k iv\. elimas, laiko praleidimas.\\
  		Derinti darbo grafika ir i\v snaudoti papildom\k a laik\k a. \\
  		Neišvengus reik\. etu priskirti papildom\k a darbuotoj\k a prie u\v zsitesusio darbo.
  		
  		
	\item Produkto kokybe blogesn\. e,	programinis nesusiderinamumas. \\ 
  		Kurti konferencijas, kuriose visi parodo k\k a \k ivykd\. e,  ir kaip. \\
  		Atskiriem darbam priskirti darbuotoj\k a, kuris sugeba komunikuoti.
  
	\item Produkto dokumentacija nepateikiama laiku.
  		Sudarin\. eti grafikus dokumento k\= urimui, atsi\v zvelgti \k i darbinink\k u nor\k a
  		j\k i kurti, sudaryti atsiskaitym\k u ta\v skus, kuriose ai\v skinamos dokumente 
  		susidurtos problemos.
  		
\end{enumerate}

\subsection{K\k a steb\. eti, atsi\v zvelgti padid\. ejus rizikai}

\begin{enumerate}
	\item Komunikavimo tr\= ukumas . \\
 		U\v zsakovo reakcij\k a \k i susitarimus, elgsen\k a nuo komunikavimo b\= udo pakeitimo. 
  		
	\item Produktas neatitinka reikalavim\k u . \\
		Kaip sprend\v ziamos problemos, kiek padaroma atsiskaitymuose. Kokie klausimai i\v skeliami konferencijose.
		
	\item Darbo vietos neu\v ztikrinimas.\\
  		Konkurencij\k a tarp darbo viet\k u, darbo privalumus, santykius tarp bendradarbi\k u,
  		kaip sprend\v ziamos i\v skilusios b\. edos esant dideliam spaudimui.
  		
	\item Nekvalifikuoti darbuotojai. \\
  		Kaip naujokai sugeba susidoroti su i\v skilusiom problemom, \\ kiek \v zmoni\k u
  		pra\v sosi \k i darb\k a ir kiek patenk\k a, ar apklausose i\v sangrin\. ejamos
  		b\= u tiniausios detal\. es. Taip pat kaip  darbuotojai supranta dokumentacij\k a.

	\item Prasta produkto kokybe, nei\v sbaigtos produkto dalys.\\
		kas vyksta su darb\k u kokyb\k e did\. ejant darb\k u spaudimui, art\. ejant \\
		projekto terminui, steb\. eti komandos nari\k u spendimus.
  		
	\item Klaid\k u \k iv\. elimas, laiko praleidimas.\\
  		Kaip sudarin\. ejamas darbo grafikas, kaip darbuotojai \k ivertina tai k\k a jiem
  		reikia padaryti.
  		
	\item Produkto kokybe blogesn\. e,	programinis nesusiderinamumas. \\ 
  		Santykius tarp bendradarbi\k u, komunikavim\k a esant grup\. ese, bendro darbo 
  		naud\k a ir atskiro darbo naud\k a.
  		
	\item Produkto dokumentacija nepateikiama laiku.
  		Steb\. eti kaip ra\v omas taisomas dokumentas, kada pristabdomas dokumento ra\v symas
  		ir kada ra\v sant dokument\k a atsiranda problemos.

	
  		
\end{enumerate}

\newpage

\subsection{Rizikos valdymas}
\begin{enumerate}
	\item Komunikavimo tr\= ukumas . \\
 			Keisti komuniavimo b\v uda, truputi pastum\. eti u\v zsakova jud\. eti pirmyn.
  			
  	\item Produktas neatitinka reikalavim\k u . \\
		Spr\k esti, keisti reikalavimus su u\v sakovu, pastum\. eti komanda dirbti toliau.
		
	\item Darbo vietos neu\v ztikrinimas.\\
  			Padaryti greit\k a apklaus\k a kas napatinka darbo vietoje, i\v snagrin\. eti\\
  			apklaus\k a ir i\v staisyti neprivalumus, padidinti ma\v zus atlyginimus
  		
	\item Nekvalifikuoti darbuotojai. \\
			Per\v ziur\. eti projekto dokument\k a \k isitikinti, kad visi projekto aspektai
			yra i\v sai\v skinami.
			Jei dokumentacija gera tai atleisti labiausiai nekvalifikuotus, kurie projekt\k a labiausiai stabdo. \\
			Kitu atveju tiesiog patobulinti dokumentacij\k a
			.
  	
	\item Prasta produkto kokybe, nei\v sbaigtos produkto dalys.\\
			Nusisamdyti \v zmogu gerai i\v smananti nei\v sbaigt\k u produkto dali\k u 
			k\= urime, per\v ziur\. eti projekto dokument\k a \k isitikinti, 
			kad visi projekto aspektai yra i\v sai\v skinami. \\ 
			Susisiekti su u\v zsakovu, keisti produkt\k a.
		
	\item Klaid\k u \k iv\. elimas, laiko praleidimas.\\
  			Per\v zvelgti darb\k u planus, \k ivertinti grafikus.
  			Jei jie geri, tai atleisti daugiausiai klaid\k u privelen\v cius, projekt\k a
  			stabda\v cius darbuotojus, nusisamdyti naujus darbuotojus. Jei ne tada \k isakyti
  			sudaryti gera darb\k u plan\k a ir kartoti veiksm\k a nuo prad\v zi\k u.
  		
	\item Produkto kokybe blogesn\. e, programinis nesusiderinamumas. \\ 
  			Nusisamdyti papildomus darbuotojus gerai nusimananti\v cius ir komunikuojan\v cius
  			ir sudaryti \k ivesti \k i sudaryta grup\k e ar sudaryti nauj\k a
  		
	\item Produkto dokumentacija nepateikiama laiku.\\
  			Nusisamdyti daugiau \v zmoni\k u ruo\v sti dokumentacij\k a.


\end{enumerate}




\end{document}
