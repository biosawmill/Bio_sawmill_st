\documentclass[a4paper,12pt]{article}
\usepackage[utf8x]{inputenc}
\usepackage[T1]{fontenc}
%\usepackage[T2A]{fontenc} % jei yra kirilica
\usepackage[hmargin={30mm,15mm},vmargin={20mm,20mm},bindingoffset=0mm]{geometry}
\usepackage[onehalfspacing]{setspace}
\usepackage[colorlinks=true, linkcolor=blue, citecolor=blue, urlcolor=blue, unicode]{hyperref}
%\parindent=7mm
\usepackage{graphicx}
\renewcommand{\refname}{Literatūros sąrašas} % article
%\renewcommand{\bibname}{Literatūros sąrašas} % report
\renewcommand{\contentsname}{Turinys}
\usepackage[T1]{fontenc} 

% Lukas paketai

\usepackage{eso-pic}
\usepackage{indentfirst}
\usepackage{setspace}
\usepackage{placeins}
\usepackage{booktabs}% http://ctan.org/pkg/booktabs
\usepackage{tabularx}% http://ctan.org/pkg/tabularx
\usepackage[parfill]{parskip}
\usepackage[unicode]{hyperref}
\usepackage{hyperref}
\usepackage{tocloft}
\usepackage{graphicx}
\newcommand\AtPageUpperRight[1]{\AtPageUpperLeft{%
   \makebox[\paperwidth][r]{#1}}}
\usepackage[dotinlabels]{titletoc}
\usepackage[capposition=top]{floatrow}
\hypersetup{
    colorlinks,
    citecolor=black,
    filecolor=black,
    linkcolor=black,
    urlcolor=black
}
\usepackage{secdot}




\begin{document}


\renewcommand{\cftdot}{.}	
\renewcommand{\cftsecleader}{\cftdotfill{\cftdotsep}}

\thispagestyle{empty} % nerasomas psl. nr


\begin{center}
 VILNIAUS UNIVERSITETAS 
 
MATEMATIKOS IR INFORMATIKOS FAKULTETAS

MATEMATINĖS INFORMATIKOS KATEDRA

\vspace{4cm}

Projekto vadovė \ \ \textbf{Aistė Čiplytė} \\
\textbf{Lukas Tutkus} \\
\textbf{Julius Daukšas} \\
\textbf{Dominykas Smaliukas} \\
\textbf{Robert Stankevič} \\

\vspace{0.2cm}

Bioinformatikos studijų programos grupė BioSawmill



\vspace{3cm}
\textbf{\Large Dvimatė pjovimo optimizacija}\\
\textbf{\Large Projekto planas}

\vfill

Vilnius \ \  2015
\end{center}



\clearpage

\tableofcontents
\clearpage
%\maketitle 
%\section*{Apžvalga}
%\addcontentsline{toc}{section}{Apžvalga} % rasoma turinyje




%\section{\centerline{„Dvimatė pjovimo optimizacija“ \\ PROJEKTO VIZIJA.}}
\setlength\parindent{24pt}

\section{ Dydžio ir pastangų vertinimas}
\subsection{Projekto dydžio apskaičiavimas}
Naudojame funkcinių taškų metriką.
\begin{enumerate}
\item \textbf{Number of external imputs}\\
6.
\item \textbf{Number of external outputs}\\
6.
\item \textbf{Number of external inquiries}\\
6.
\item \textbf{Number of internal logical files}\\
15.
\item \textbf{Number of external interface files}\\
1.
\end{enumerate}
Tikslesniam dyžio matavimui naudojame papildomus faktorius.
\begin{enumerate}
\item \textbf{Does the system require reliable backup and recovery?}\\
4.
\item \textbf{Are data communications required?}\\
2.
\item \textbf{ Are there distributed processing functions?}\\
3.
\item \textbf{Is performance critical?}\\
4.
\item \textbf{Will the system run in a existing, heavily utilized operational environment?}\\
0.
\item \textbf{Does the system require on-line data entry?}\\
3.
\item \textbf{Does the on-line data entry require the input transaction to be built over multiple screens or operations?}\\
3.
\item \textbf{Are the master files updated on-line?}\\
2.
\item \textbf{Are the inputs, outputs, files or inquiries complex?}\\
2.
\item \textbf{Is the internal processing complex?}\\
3.
\item \textbf{Is the code designed to be reusable?}\\
0.
\item \textbf{Are conversion and installation included in the design?}\\
1.
\item \textbf{Is the system designed for multiple installations in different organizations?}\\
0.
\item \textbf{Is the application designed to facilitate change and ease of use by the user?}\\
1.
\end{enumerate}
\textbf{VISO: 28}\\\\
Apskaičiuojame funkcinius taškus pagal formulę
$$ FP = count\ total \times [0.65 + 0.01 \times \sum(F_i)] $$
$FP$ - projekto dydžio vertinimas funkcinias taškais.

 \begin{center}
  \begin{tabular}{ l | l | l || r }
    \hline
    Domain & Est. & Weight & FP \\ \hline
    External inputs  & 6 & 4 & 24 \\ \hline
    External outputs  & 6 & 4 & 24 \\ \hline
    External inquiries & 6 & 4 & 15 \\ \hline
    Internal logical files & 15 & 7 & 105 \\ \hline
    External interface files & 1 & 10 & 10 \\ \hline
    \multicolumn{3}{ l }{Total:} & 187 \\
    \hline
  \end{tabular}
\end{center}

	 \large\textbf{$$FP \ = \ 187 \ \times \ 0.93 \ =\  174 $$}
\subsection{Pastangų vertinimas}
\begin{frame}
\centering
\hspace{2cm}
\label{my-label}
\begin{tabular}{|c|c|}
\hline
\textbf{Veikla}                                                                & \textbf{\begin{tabular}[c]{@{}c@{}}Darbo valandų\end{tabular}} \\ \hline
Analizė                                                                        & 30                                                               \\ \hline
\begin{tabular}[c]{@{}c@{}}Darbo aplinkos ruošimas ir priežiūra\end{tabular} & 36                                                               \\ \hline
\begin{tabular}[c]{@{}c@{}}Dokumentacijos pildymas\end{tabular}              & 70                                                               \\ \hline
Palaikymas                                                                     & 60                                                               \\ \hline
Programavimas                                                                  & 328 \\ \hline
Projektavimas                                                                  & 100                                                              \\ \hline
Projekto valdymas                                                              & 100                                                              \\ \hline
Svetainės dizaino kūrimas ir palaikymas                                         & 44                                                               \\ \hline
Testavimas                                                                     & 72                                                               \\ \hline
VISO &
840
\\ \hline
\end{tabular}
\end{frame}


\section{ Tvarkaraštis}


 Tvarkarastis pateikiamas kitame faile (gant.html).


\section{Konfiguracin\. e kontrol\. e}
Kekvieno produkto gyvavimo cikle atsiranda nei\v sviangiam\k u pasikeitim\k u.\\
Konfiguracin\. e kontrol\. e u\v zikrina, kad tie pasikeitimai b\= utu minimal\= us.
\v Siame skyriuje aptarsiu galimus pasikeitimai i\v s u\v zakovo pus\. es bei kaip juos kontroliuoti. \\
U\v zsakovas gali nuspr\k esti
\begin{itemize}
	\item pakeisti produkto informacij\k a.
	\item kada ir kiek i\v spl\. esti, pagreitinti programos funkcionalum\k a.
	\item koki\k i servis\k u papildymo tr\= uksta.
\end{itemize}

Kekvin\k a u\v zsakovo sprendim\k a turime ivykdyti. Problema tame, kad reikia atsekti kada ir kur ivykdyti pakeitim\k a.\\
Problemos sprendimas ir b\= utu konfiguracin\. es kontrol\. es naudojimas.\\
Ji u\v ztikrina, kad informacija saugojama, kontroliuojama sistemos k\= urimo, keitimo ir testavimo metu ir taip pat ar atitinka vykdomus \v reikalavimus. \\
Taip pat u\v ztikrinti kad informacij\k a apie projekta laisvai pasiekiama.\\
\large\textbf{Konfiguracin\k e kontrol\. es u\v zuduotys}
\begin{enumerate}
	\item Rasti kokios projekto dalys gali keistis.
	\item kontroliuoti vien\k a ar kelis rastus.
	\item lengvinti skirting\k u versij\k u aplikacij\k u konstrukcij\k a. 
	\item u\v ztikrinti kad projekto kokybe palaikoma, kai konfiguracija atnaujinama.
\end{enumerate}
\hfill \\
Konfiguracijos kontrole u\v ztikrina programos konfiguracijos element\k u identifikavim\k a, pasikeitimo kontrol\k e, versijos kontrol\k e, konfiguracijos audit\k a ir ataskaita tiek projekto k\= urimo metu, tiek palaikymo metu.\\
\large\textbf{Projekte kontrol\. e u\v ztikrinama}
\begin{enumerate}
	\item Programos dokumentacijai.
	\item Programos analiz\. es failai.
	\item Programos diegimo failam.
	\item Programavimo kodą.
	\item Internetinės svetainės kodas.
\end{enumerate}
\subsection{Versij\k u kontrol\. e}
Versij\k u kontrolei naudojamas "Github".
Ne tik programos kodui versijonuoti bet taip pat ir apie tai informuoti bendradarbius.

\large\textbf{Joje laikomi failai:}
\normalsize
\begin{enumerate}

	\item Vadyba ( organizacijos strukt\= ura, architekt\= uros informacija ).
	\item Modeliavmas ( analiz\. e, dizainas ).
	\item Konstrukcijos ( \v saltinio kodas, kompiliavimo instrukcijos )
	\item Testai ( skriptai, rezultatai, kokyb?s matavimai ).
	\item Dokumentai (projekto, dizaino dokumentai, naudojimo \v zinynas ).
	\item Projekto valdymas ( projekto trukm\. e, tvarkara\v stis, auditas ).
		
\end{enumerate}
\clearpage 


\section{Pakeitimų valdymo procedūra}

\large\textbf{Pakeitimų skirstymas}
\begin{enumerate}
	\item funkcijos arba aplikacijos ar rezultat? atvaizdavimo patobulinimai ar klaid\k u juose pataisymas
	\item funkcijos ar atvaizdavimo patobulinimai, kurie turi ?takos kitoms projekto programoms
	\item funkcij\k u ir naudojam\k u duomen\k u pakeitimai, kurie stipriai pakei\v cia, patobulina ar supaprastina web aplikacij\k u ar rezultat\k u atvaizdavim\k a
	\item dideli dizaino ar navigacijos web aplikacijoje pakeitimai, kurie stipriai \k itakoti vartotoj\k a.
\end{enumerate}

Šioje pastraipoje yra aprašomas algoritmas, kaip bus veikiama, iškylus poreikiui atlikti pakeitimus. 
Susidūrus su noru kažką pakeisti, jam, pirma, bus priskirta apimties/svarbumo klasė. Nepriklausomai nuo priskirtos klasės, jeigu pakeitimui įgyvendinti nėra nepritariama, tai jis bus pradedamas vykdyti pažingsniui: 
\begin{enumerate}
	\item taisytinas objektas (programos kodas) bus išsikeliamas
 (check-out)
 	\item atliekami pakeitimai
 	\item ištestuojami pakeitimai
 	\item pakeitimai įkeliami į repozitoriją (check-in)
 	\item paleidžiama web aplikacija su atliktu pakeitimu.
\end{enumerate}
\clearpage
Priklausomai nuo priskirtos klasės, dar prieš išsikeliant taisytinus objektus, atliekamos tokios procedūros: 
\begin{itemize}
	\item 2-os kl. pakeitimui atilikti tikrinami visi objektai, kurie susiję su keistinu objektu.
		Įvertinama, kokią įtaką jiems tie pakeitimai turės. kiek tuomet pakeitimų (derinimų) reikės atlikti šiuose susijusiuose objektuose.		
	\item 3-os kl. pakeitimui atlikti rašomas glaudus pakeitimo aprašymas.\\
	Pateikiamas visiems projekto vykdytojams, kad šie susipažintų ir pritartų, ir nesant jų pritarimui, apie pakeitimą diskutuojama ieškant kompromiso arba jis visiškai atmetamas.
	\item 4-os kl. pakeitimui atlikti rašomas glaudus pakeitimo aprašymas.\\
	Pateikiamas visiems užsakovams, kad šie peržiūrėtų ir pritartų, o nesant pritarimui, klausomasi jų siūlymų, ieškoma abi puses patenkinančio susitarimo ir jam susiformavus, pakeitimas atliekamas. 
\end{itemize}
 
 \clearpage
 
\section{Resursų planas}



Žmones esantys kompanijoje, darbuotojai.


\small
\begin{frame}
\centering
\hspace{-3.5cm}
\label{my-label}
\begin{tabular}{|c|c|l|c|}
\hline
\textbf{\begin{tabular}[c]{@{}c@{}}Darbuotojas/\\ Pareigos\end{tabular}} & \textbf{\begin{tabular}[c]{@{}c@{}}Darbuotojų\\ kiekis\end{tabular}} & \multicolumn{1}{c|}{\textbf{Darbo pobūdis}}                                                                                                                                                        & \textbf{\begin{tabular}[c]{@{}c@{}}Darbo\\ valandų\footnotemark[1]\end{tabular}} \\ \hline
Administratorius                                                         & 1                                                                    & \begin{tabular}[c]{@{}l@{}}Sprendžia iškilusias techninės ir programinės \\ įrangos problemas. Atsakingas už darbuotojams \\ reikalingų resursų pasiekiamumą.\end{tabular}                             & 36 \\ \hline
Analitikas                                                               & 1                                                                    & \begin{tabular}[c]{@{}l@{}}Tikrina ar programuotojų sukurtas sistemos \\ funkcionalumas atitinka kliento lūkesčius. Iš dalies \\atlieka testuotojo vaidmenį.  Ruošia dokumentaciją.\end{tabular}    & 100                                                              \\ \hline
Dizaineris                                                               & 1                                                                    & \begin{tabular}[c]{@{}l@{}}Kuria pradinį svetainės dizainą, o projekto eigoje keičia \\ dizainą pagal kliento reikalavimus. Suteikia profesionalias \\ konsultacijas programuotojams.\end{tabular} & 44                                                               \\ \hline
Programuotojas-testuotojas                                               & 2                                                                    & Rašo programinį kodą, o vėliau jį testuoja.                                                                                                                                                        & 120x2                                                            \\ \hline
Programuotojas-testuotojas                                               & 1                                                                    & \begin{tabular}[c]{@{}l@{}}Rašo programinį kodą, o vėliau jį testuoja. Paskutiniame \\ etape bus atsakingas už produkto palaikymo užtikrinimą.\end{tabular}                                         & 140                                                              \\ \hline
Projekto vadovas                                                         & 1                                                                    & \begin{tabular}[c]{@{}l@{}}Pagrindinis asmuo bendraujantis su klientu. Atsakingas už\\ projekto vystymą, komandos narių parinkimą. Konsultuoja\\ komandos narius.\end{tabular}                     & 100                                                              \\ \hline
Sistemos inžinierius                                                     & 1                                                                    & \begin{tabular}[c]{@{}l@{}}Pagal pateiktus ir išanalizuotus kliento poreikius sudaro \\ sistemos architektūrą (įskaitant ir techninę dalį). \\ Konsultuoja programuotojus.\end{tabular}                & 100                                                              \\ \hline
\end{tabular}
\end{frame}
\vspace{1cm}
\\
\textbf{Pastaba}: laikoma, kad žmogus dirbantis nepilnu etatu, tuo pačiu metu dirba prie kitų įmonės projektų.
\footnotetext[1]{Darbo valandų skaičius apskaičiuotas atsižvelgus į darbuotojų numatytas darbo valandas kiekviename etape.}

Techninės įrangos poreikiai:
\begin{itemize}
\item Kekvienam darbuotojui suteikiamas asmeninis kompiuteris, kuriame sudiegta reikalinga programinė įranga.
\end{itemize}

\clearpage

\section{Rizikos valdymas}

\subsection{Identifikavimas}



\FloatBarrier
%\begin{figure}[H]

\begin{frame}
\centering
\hspace*{-3 cm}
%\caption{Rizik\k u valdymas}
\small
\setlength\tabcolsep{3pt}
\begin{tabular}{|l|l|l|l|l|l|}
		\hline
 		Nr. & Rizik\k u prie\v zastys & Rizikos & Kategorija & Tikimyb\. e
 		& \k Itaka  \\ \hline
 		1. &
 		\begin{tabular}[c]{@{}l@{}}
			Kalbos barjeras, \\
			u\v zsakovo laiko tr\= ukumas
 		\end{tabular} &
 		\begin{tabular}[c]{@{}l@{}}
			Komunikavimo tr\= ukumas
 		\end{tabular} & 
 		PR 	 & 
 		didel\. e & 
 		1	\\ \hline
 
 		2. & 			
 		\begin{tabular}[c]{@{}l@{}}
			N\. era atsiskaitymo ta\v sk\k u,\\
			ma\v zai pasitarim\k u
 		\end{tabular}  & 
 		Produktas neatitinka reikalavim\k u &
 		VR	 & 
 		didel\. e &
 		2	\\ \hline
 		
 		3. &
 		\begin{tabular}[c]{@{}l@{}}
			Nepateikiami testinai duomenys
 		\end{tabular} &
 		\begin{tabular}[c]{@{}l@{}}
			P.O. algoritmas nekokybiškas
 		\end{tabular} & 
 		PR 	 & 
 		vidutinė & 
 		1	\\ \hline
 		
 		4. &
 		\begin{tabular}[c]{@{}l@{}} 
 			Neu\v ztikrinamos geros \\darbo s\k alygos \\ 
 		\end{tabular} &
		\begin{tabular}[c]{@{}l@{}} 
 			Darbuotojai palieka darbovietę, \\ 
 			dažnas darbuotoj\k u keitimas, \\
 			produkto v\. elinimas, \\
 			mai\v sytas produktas \\
 		\end{tabular} &
  		PR 	 & 
 		didel\. e& 
 		3	\\ \hline
 
 		5. &
 		\begin{tabular}[c]{@{}l@{}} 
 			Trumpos apklausos, ma\v zai\\
 			testavimo priimant\\
 			naujus darbuotojus 
 		\end{tabular} &
 		
 		\begin{tabular}[c]{@{}l@{}} 
			Nekvalifikuoti darbuotojai 			
  		\end{tabular} & 
 		DR 	 & 
 		vidutin\. e & 
 		2	\\ \hline
 		
 		6. &
 		\begin{tabular}[c]{@{}l@{}}
			Galimybi\k u pervertinimas,  \\
			neai\v skus darbo planas.
 		\end{tabular} &
 		
 		\begin{tabular}[c]{@{}l@{}}	
 			Produkto kokyb\. e pablog\. eja, \\
 			nei\v sbaigtos produkto dalys\\ 
 		\end{tabular} &
 		PR 	 & 
 		ma\v za& 
 		1 \\ \hline
 		
 		7. &
 		\begin{tabular}[c]{@{}l@{}} 
 			Darbas vir\v svaland\v ziais, \\
 			blogas darbų planas.
 		\end{tabular} &
		\begin{tabular}[c]{@{}l@{}} 
 			Klaid\k u \k ivėlimas, \\
 			laiko praleidimas
 		\end{tabular} &
  		PR 	 & 
 		vidutin\. e& 
 		3	\\ \hline
 			
 		8. &	 
 		Darbuotoj\k u nekomunikavimas & 
		\begin{tabular}[c]{@{}l@{}}
			Produkto kokybė blogesn\. e,\\
 			programinis nesuderinamumas \\
		\end{tabular}  & 
 		PR 	 & 
 		ma\v za & 
 		3	\\ \hline
 		
 		9. &
 		\begin{tabular}[c]{@{}l@{}}
			Atsiskaitymo data atkeliama \\
			anks\v ciau
 		\end{tabular}  & 
 		V\. elavimas &
 		VR	 & 
 		ma\v za &
 		3	\\ \hline
 		
 		10. &
 		\begin{tabular}[c]{@{}l@{}}
			Dokumento ra\v symo atidėjimai.
 		\end{tabular}  & 
 		Produkto dokumentacija nepateikiama laiku &
 		VR	 & 
 		ma\v za&
 		3	\\ \hline
 
 		
	\end{tabular}%
\end{frame}\\ 

\small

\k Itakos dydžiai: 
		1 - Katastrofinis,
		2 - Kritikalus,
		3 - Ne\v zymus.\\
		PR - Produkto Rizika. DR - Darbuotoj\k u Rizika. VR - Verslo Rizika.
	
\subsection{Valdymo strategija}
Identifikave galimas rizik\k u prie\v zastis paruo\v siame strategiją, kaip jas valdysim.
\begin{itemize}
  \item Rizikos suma\v zinimas.
  \item Rizikos steb\. ejimas.
  \item Rizikos valdymas, nenumatyt\k u atv\. ej\k u planavimas.
\end{itemize}

\subsection{Rizikos suma\v zinimas, }
Stengiamasi suma\v zinti projekto valdymo galioje esan\v cias rizikas.

\begin{enumerate}

	\item Komunikavimo tr\= ukumas . \\
  		Svarbu kalb\. eti, klausti, susitarti d\. el vietos ir laiko susitikimam.\\
  		Sprendimas  b\= utų surasti \v zmog\k u, kuris gal\. etų sukomunikuoti \\
  		arba pasi\= ulyti \v kitą komunikavimo b\= udą pvz.: komunikuoti internetu. 
  		
  	\item Produktas neatitinka reikalavim\k u . \\
  		Sudaryti kontrolinius ta\v skus, atsiskaitymus, kuriuose kekvienas \\ 
  		darbuotojas ar j\k u grup\. e parašo kiek, nuveik\. e.\\
  		Daryti konferencijas, kuriose spren\v ziamos atsiradusios problemos.
  		
  	\item P.O. algoritmas nekokybiškas \\
  	Komunikuoti su užsakovu siekant tikslesnių duomenų.
  	
	\item Darbo vietos neu\v ztikrinimas.\\
  		U\v ztiktinti geras darbo sąlygas. Dažniau daryti apklausas apie darb\k a,\\ 
  		\k ivertinti iš naujo darbuotoj\k u atlyginimus.\\
  		Tam ivykus geriausia b\= utu pasamdyti atitinkam\k a darbuotoj\k a ir u\v ztikrinti,\\
  		kad nepasikartot\k u senojo darbuotojo i\v s\. ejimo prie\v zastis, jei \k imanoma.
  		
	\item Nekvalifikuoti darbuotojai. \\
  		Tinkama\k u apklaus\k u u\v ztikrinimas, įrodan\v ci\k u \k idarbinamojo sugeb\. ejimus. \\ Per\v zvelgti projekto dokumentaciją, i\v staisyti netikslumus.\\
  		Gavus nekvalifikuot\k a darbuotoj\k a, svarbu suteikti reikaling\k a asmen\k i, \\
  		kuris suteiktų jam reikalingą informaciją ir pri\v ziur\. etų darbus.

	\item Prasta produkto kokybė, nei\v sbaigtos produkto dalys.\\
		\k ivertinti projekto sud\. etingum\k a, nustatyti ar atitinka galimybes.\\
		Per v\. elai  \k ivertinus b\= utina prane\v sti u\v zsakovui ir ie\v skoti bendro 				sprendimo.
  		
	\item Klaid\k u \k iv\. elimas, laiko praleidimas.\\
  		Derinti darbo grafiką ir i\v snaudoti papildom\k a laik\k a. \\
  		Neišvengus reik\. etų priskirti papildom\k a darbuotoj\k a prie u\v zsitęsusio darbo.
  		
  		
	\item Produkto kokybė blogesn\. e,	programinis nesusiderinamumas. \\ 
  		Kurti konferencijas, kuriose visi parodo k\k a \k ivykd\. e,  ir kaip. \\
  		Atskiriem darbam priskirti darbuotoj\k a, kuris sugeba komunikuoti.
  
	\item Produkto dokumentacija nepateikiama laiku.
  		Sudarin\. eti grafikus dokumento k\= urimui, atsi\v zvelgti \k i darbinink\k u nor\k a
  		j\k i kurti, sudaryti atsiskaitym\k u ta\v skus, kuriose ai\v skinamos dokumente 
  		susidurtos problemos.
  		
\end{enumerate}

\clearpage

\subsection{K\k a steb\. eti, atsi\v zvelgti padid\. ejus rizikai}

\begin{enumerate}
	\item Komunikavimo tr\= ukumas . \\
 		U\v zsakovo reakcij\k a \k i susitarimus, elgsen\k a nuo komunikavimo b\= udo pakeitimo. 
  		
	\item Produktas neatitinka reikalavim\k u . \\
		Kaip sprend\v ziamos problemos, kiek padaroma atsiskaitymuose. Kokie klausimai i\v skeliami konferencijose.
		
	\item P.O. algoritmas nekokybiškas\\
  	Komunikuoti su užsakovu siekant tikslesnių duomenų.
  		
	\item Darbo vietos neu\v ztikrinimas.\\
  		Konkurencij\k a tarp darbo viet\k u, darbo privalumus, santykius tarp bendradarbi\k u,
  		kaip sprend\v ziamos i\v skilusios b\. edos esant dideliam spaudimui.
  		
	\item Nekvalifikuoti darbuotojai. \\
  		Kaip naujokai sugeba susidoroti su i\v skilusiom problemom, \\ kiek \v zmoni\k u
  		pra\v sosi \k i darb\k a ir kiek patenka, ar apklausose i\v sangrin\. ejamos
  		būtiniausios detal\. es. Taip pat, kaip  darbuotojai supranta dokumentacij\k a.

	\item Prasta produkto kokybė, nei\v sbaigtos produkto dalys.\\
		Kas vyksta su darb\k u kokybe did\. ejant darb\k u spaudimui, art\. ejant \\
		projekto terminui, steb\. eti komandos nari\k u spendimus.
  		
	\item Klaid\k u \k iv\. elimas, laiko praleidimas.\\
  		Kaip sudarin\. ejamas darbo grafikas, kaip darbuotojai \k ivertina tai, k\k a jiem
  		reikia padaryti.
  		
	\item Produkto kokybė blogesn\. e,	programinis nesusiderinamumas. \\ 
  		Santykius tarp bendradarbi\k u, komunikavim\k a esant grup\. ese, bendro darbo 
  		naud\k a ir atskiro darbo naud\k a.
  		
	\item Produkto dokumentacija nepateikiama laiku.
  		Steb\. eti kaip ra\v omas taisomas dokumentas, kada pristabdomas dokumento ra\v symas
  		ir kada, ra\v sant dokument\k a, atsiranda problemos.

	
  		
\end{enumerate}

\newpage

\subsection{Rizikos valdymas}
\begin{enumerate}
	\item Komunikavimo tr\= ukumas . \\
 			Keisti komuniavimo būdą, truputi pastum\. eti u\v zsakovą jud\. eti pirmyn.
  			
  	\item Produktas neatitinka reikalavim\k u . \\
		Spr\k esti, keisti reikalavimus su u\v sakovu, pastum\. eti komandą dirbti toliau.
		
	\item P.O. algoritmas nekokybiškas. \\
			Projekto vadovas autorizuojamas leisti atsargines lėšas generuojant duomenis
		
	\item Darbo vietos neu\v ztikrinimas.\\
  			Padaryti greit\k a apklaus\k a, kas napatinka darbo vietoje, i\v snagrin\. eti\\
  			apklaus\k a ir i\v staisyti trūkumus, padidinti ma\v zus atlyginimus
  		
	\item Nekvalifikuoti darbuotojai. \\
			Per\v ziur\. eti projekto dokument\k a \k isitikinti, kad visi projekto aspektai
			yra i\v sai\v skinami.
			Jei dokumentacija gera, tai atleisti labiausiai nekvalifikuotus, kurie projekt\k a labiausiai stabdo. \\
			Kitu atveju, tiesiog, patobulinti dokumentacij\k a
			.
  	
	\item Prasta produkto kokybė, nei\v sbaigtos produkto dalys.\\
			Nusisamdyti \v zmogų, gerai i\v smanantį nei\v sbaigt\k u produkto dali\k u 
			k\= urimą, per\v ziur\. eti projekto dokumentą, isitikinti, 
			kad visi projekto aspektai yra i\v sai\v skinami. \\ 
			Susisiekti su u\v zsakovu, keisti produkt\k a.
		
	\item Klaid\k u \k iv\. elimas, laiko praleidimas.\\
  			Per\v zvelgti darb\k u planus, \k ivertinti grafikus.
  			Jei jie geri, tai atleisti daugiausiai klaid\k u privelen\v cius, projekt\k a
  			stabda\v cius darbuotojus, nusisamdyti naujus darbuotojus. Jei ne, tada \k isakyti
  			sudaryti gera darb\k u plan\k a ir kartoti veiksm\k a nuo prad\v zi\k u.
  		
	\item Produkto kokybė blogesn\. e, programinis nesusiderinamumas. \\ 
  			Nusisamdyti papildomus darbuotojus gerai nusimanan\v cius ir komunikuojan\v cius
  			ir įvest \k i sudarytą grup\k e ar sudaryti nauj\k a
  		
	\item Produkto dokumentacija nepateikiama laiku.\\
  			Nusisamdyti daugiau \v zmoni\k u ruo\v sti dokumentacij\k a.


\end{enumerate}




\end{document}
