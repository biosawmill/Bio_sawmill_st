\documentclass[12pt]{article}
\usepackage[utf8x]{inputenc} % atkomentuoti, jei naudojat utf-8 koduot� 
\usepackage[L7x]{fontenc}
\usepackage[parfill]{parskip}
\usepackage[lithuanian]{babel}
\usepackage[unicode]{hyperref} % naudojant ?d?ti dar backend'?
                                % opcijas: dvips | dvipdfm | pdftex

\begin{document}
\section{Konfiguracin\. e kontrol\. e}
Kekvieno produkto gyvavimo cikle atsiranda nei\v sviangiam\k u pasikeitim\k u.\\
Konfiguracin\. e kontrol\. e u\v zikrina, kad tie pasikeitimai b\= utu minimal\= us.
\v Siame skyriuje aptarsiu galimus pasikeitimai i\v s u\v zakovo pus\. es bei kaip juos kontroliuoti. \\
\\
U\v zsakovas gali nuspr\k esti
\begin{itemize}
	\item pakeisti produkto informacij\k a.
	\item kada ir kiek i\v spl\. esti, pagreitinti programos funkcionalum\k a.
	\item koki? servis\k u papildymo tr\= uksta.
\end{itemize}

Kekvin\k a u\v zsakovo sprendim\k a turime ivykdyti. Problema tame, kad reikia atsekti kada ir kur ivykdyti pakeitim\k a.\\
Problemos sprendimas ir b\= utu konfiguracin\. es kontrol\. es naudojimas.\\
Ji u\v ztikrina, kad informacija saugojama, kontroliuojama sistemos k\= urimo, keitimo ir testavimo metu ir taip pat ar atitinka vykdomus \v reikalavimus. \\
Taip pat u\v ztikrinti kad informacij\k a apie projekta laisvai pasiekiama.\\
Konfiguracin\k e kontrol\. es u\v zuduotys:
\begin{enumerate}
	\item Rasti kokios projekto dalys gali keistis.
	\item kontroliuoti vien\k a ar kelis rastus.
	\item lengvinti skirting\k u versij\k u aplikacij\k u konstrukcij\k a. 
	\item u\v ztikrinti kad projekto kokybe palaikoma, kai konfiguracija atnaujinama.
\end{enumerate}
\newpage
Konfiguracijos kontrole u\v ztikrina programos konfiguracijos element\k u identifikavim\k a, pasikeitimo kontrol\k e, versijos kontrol\k e, konfiguracijos audit\k a ir ataskaita tiek projekto k\= urimo metu, tiek palaikymo metu.\\\\


\large\textbf{Projekte kontrol\. e u\v ztikrinama}
\normalsize
\begin{enumerate}
	\item Programos dokumentacijai.
	\item Programos analiz\. es failai.
	\item Programos diegimo failam.
	\item Sistemos failam.
\end{enumerate}

\

\section{Versij\k u kontrol\. e}

Versij\k u kontrolei naudojamas "Github".\\
Ne tik programos kodui versijonuoti bet taip pat ir apie tai informuoti bendradarbius.\\

\large\textbf{Joje laikomi failai:}
\normalsize
\begin{enumerate}

	\item Vadyba ( organizacijos strukt\= ura, architekt\= uros informacija ).
	\item Modeliavmas ( analiz\. e, dizainas ).
	\item Konstrukcijos ( \v saltinio kodas, kompiliavimo instrukcijos )
	\item Testai ( skriptai, rezultatai, kokyb?s matavimai ).
	\item Dokumentai (projekto, dizaino dokumentai, naudojimo \v zinynas ).
	\item Projekto valdymas ( projekto trukm\. e, tvarkara\v stis, auditas ).
		
\end{enumerate}


\end{document}