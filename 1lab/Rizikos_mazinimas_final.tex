\documentclass[12pt]{article}
\usepackage[T1]{fontenc} 
\usepackage[utf8]{inputenc}
\usepackage{float}
\usepackage{placeins}
\usepackage{booktabs}% http://ctan.org/pkg/booktabs
\usepackage{tabularx}% http://ctan.org/pkg/tabularx

\begin{document}

\section{Rizikos valdymas}

\subsection{Identifikavimas}



\FloatBarrier
%\begin{figure}[H]

\begin{frame}
\centering
\hspace*{-3 cm}
%\caption{Rizik\k u valdymas}
\small
\setlength\tabcolsep{3pt}
\begin{tabular}{|l|l|l|l|l|l|}
		\hline
 		Nr. & Rizik\k u prie\v zastys & Rizikos & Kategorija & tikimyb\. e
 		& \k Itaka  \\ \hline
 		1. &
 		\begin{tabular}[c]{@{}l@{}}
			Kalbos barjeras, \\
			u\v zsakovo laiko tr\= ukumas
 		\end{tabular} &
 		\begin{tabular}[c]{@{}l@{}}
			Komunikavimo tr\= ukumas
 		\end{tabular} & 
 		PR 	 & 
 		didel\. e & 
 		1	\\ \hline
 
 		2. & 			
 		\begin{tabular}[c]{@{}l@{}}
			N\. era atsiskaitymo ta\v sk\k u,\\
			ma\v zai pasitarim\k u
 		\end{tabular}  & 
 		Produktas neatitinka reikalavim\k u &
 		VR	 & 
 		didel\. e &
 		2	\\ \hline
 		
 		3. &
 		\begin{tabular}[c]{@{}l@{}} 
 			Neu\v ztikrinamos geros \\darbo s\k alygos \\ 
 		\end{tabular} &
		\begin{tabular}[c]{@{}l@{}} 
 			Darbuotojai palieka darboviet\. e, \\ 
 			dažnas darbuotoj\k u keitimas, \\
 			produkto v\. elinimas \\
 			mai\v sytas produktas \\
 		\end{tabular} &
  		PR 	 & 
 		didel\. e& 
 		3	\\ \hline
 
 		4. &
 		\begin{tabular}[c]{@{}l@{}} 
 			Trumpos apklausos, ma\v zai\\
 			testavimo priimant\\
 			naujus darbuotojus 
 		\end{tabular} &
 		
 		\begin{tabular}[c]{@{}l@{}} 
			Nekvalifikuoti darbuotojai 			
  		\end{tabular} & 
 		DR 	 & 
 		vidutin\. e & 
 		2	\\ \hline
 		
 		5. &
 		\begin{tabular}[c]{@{}l@{}}
			Galimybi\k u pervertimas  \\
			Neai\v skus darbo planas.
 		\end{tabular} &
 		
 		\begin{tabular}[c]{@{}l@{}}	
 			Produkto kokyb\. e pablog\. eja \\
 			nei\v sbaigtos produkto dalys\\ 
 		\end{tabular} &
 		PR 	 & 
 		ma\v za& 
 		1 \\ \hline
 		
 		6. &
 		\begin{tabular}[c]{@{}l@{}} 
 			Darbas vir\v svaland\v ziais, \\
 			blogas darbu planas.
 		\end{tabular} &
		\begin{tabular}[c]{@{}l@{}} 
 			Klaid\k u \k ivelimas, \\
 			Laiko praleidimas
 		\end{tabular} &
  		PR 	 & 
 		vidutin\. e& 
 		3	\\ \hline
 			
 		7. &	 
 		Darbuotoj\k u nekomunikavimas & 
		\begin{tabular}[c]{@{}l@{}}
			Produkto kokybe blogesn\. e,\\
 			programinis nesuderinamumas \\
		\end{tabular}  & 
 		PR 	 & 
 		ma\v za & 
 		3	\\ \hline
 		
 		8. &
 		\begin{tabular}[c]{@{}l@{}}
			Atsiskaitymo data atkeliama \\
			anks\v ciau
 		\end{tabular}  & 
 		V\. elavimas &
 		VR	 & 
 		ma\v za &
 		3	\\ \hline
 		
 		9. &
 		\begin{tabular}[c]{@{}l@{}}
			Dokumento ra\v symo atidejimai.
 		\end{tabular}  & 
 		Produkto dokumentacija nepateikiama laiku &
 		VR	 & 
 		ma\v za&
 		3	\\ \hline
 
 		
	\end{tabular}%
\end{frame}
	\\\\ \k Itakos dydžiai: 
		1 - Katastrofinis,
		2 - Kritikalus,
		3 - Ne\v zymus.\\
		PR - Produkto Rizika. DR - Darbuotoj\k u Rizika. VR - Verslo rizika. \\\\
	
\subsection{Valdymo strategija}
Identifikave galimas rizik\k u prie\v zastis paruo\v siame strategija, kaip jas valdysim.
\begin{itemize}
  \item Rizikos suma\v zinimas.
  \item Rizikos steb\. ejimas.
  \item Rizikos valdymas, nenumatyt\k u atv\. ej\k u planavimas.
\end{itemize}

\newpage

\subsection{Rizikos suma\v zinimas, }
Stengiamasi suma\v zinti projekto valdymo galioje esan\v cias rizikas.

\begin{enumerate}

	\item Komunikavimo tr\= ukumas . \\
  		Svarbu kalb\. eti, klausti, susitarti d\. el vietos ir laiko susitikimam.\\
  		Sprendimas variantas b\= utu surasti \v zmog\k u, kuris gal\. etu sukomunikuoti \\
  		arba pasi\= ulyti \v kita komunikavimo b\= uda pvz.: komunikuoti internetu. 
  		
  	\item Produktas neatitinka reikalavim\k u . \\
  		Sudaryti kontrolinius ta\v skus, atsiskaitymus, kuriuose kekvienas \\ 
  		darbuotojas ar j\k u grup\. e parašo kiek nuveik\. e.\\
  		Daryti konferencijas, kuriose spren\v ziamos atsiradusios problemos.
  		
	\item Darbo vietos neu\v ztikrinimas.\\
  		U\v ztiktinti geras darbo salygas. Dažniau daryti apklausas apie darb\k a,\\ 
  		\k ivertinti iš naujo darbuotoj\k u atlyginimus.\\
  		Tam ivykus geriausia b\= utu pasamdyti atitinkam\k a darbuotoj\k a ir u\v ztikrinti,\\
  		kad nepasikartot\k u senojo darbuotojo i\v s\. ejimo prie\v zastis, jei \k imanoma.
  		
	\item Nekvalifikuoti darbuotojai. \\
  		Tinkama\k u apklaus\k u u\v ztikrinimas, irodan\v ci\k u \k idarbinamojo sugeb\. ejimus. \\ Per\v zvelgti projekto dokumentacija i\v staisyti netikslumus.\\
  		Gavus nekvalifikuot\k a darbuotoj\k a, svarbu suteikti reikaling\k a asmen\k i, \\
  		kuris suteiktu jam reikalinga informacija ir pri\v ziur\. etu darbus.

	\item Prasta produkto kokybe, nei\v sbaigtos produkto dalys.\\
		\k ivertinti projekto sud\. etingum\k a, nustatyti ar atitinka galimybes.\\
		Per v\. elai  \k ivertinus b\= utina prane\v sti u\v zsakovui ir ie\v skoti bendro 				sprendimo.
  		
	\item Klaid\k u \k iv\. elimas, laiko praleidimas.\\
  		Derinti darbo grafika ir i\v snaudoti papildom\k a laik\k a. \\
  		Neišvengus reik\. etu priskirti papildom\k a darbuotoj\k a prie u\v zsitesusio darbo.
  		
  		
	\item Produkto kokybe blogesn\. e,	programinis nesusiderinamumas. \\ 
  		Kurti konferencijas, kuriose visi parodo k\k a \k ivykd\. e,  ir kaip. \\
  		Atskiriem darbam priskirti darbuotoj\k a, kuris sugeba komunikuoti.
  
	\item Produkto dokumentacija nepateikiama laiku.
  		Sudarin\. eti grafikus dokumento k\= urimui, atsi\v zvelgti \k i darbinink\k u nor\k a
  		j\k i kurti, sudaryti atsiskaitym\k u ta\v skus, kuriose ai\v skinamos dokumente 
  		susidurtos problemos.
  		
\end{enumerate}

\subsection{K\k a steb\. eti, atsi\v zvelgti padid\. ejus rizikai}

\begin{enumerate}
	\item Komunikavimo tr\= ukumas . \\
 		U\v zsakovo reakcij\k a \k i susitarimus, elgsen\k a nuo komunikavimo b\= udo pakeitimo. 
  		
	\item Produktas neatitinka reikalavim\k u . \\
		Kaip sprend\v ziamos problemos, kiek padaroma atsiskaitymuose. Kokie klausimai i\v skeliami konferencijose.
		
	\item Darbo vietos neu\v ztikrinimas.\\
  		Konkurencij\k a tarp darbo viet\k u, darbo privalumus, santykius tarp bendradarbi\k u,
  		kaip sprend\v ziamos i\v skilusios b\. edos esant dideliam spaudimui.
  		
	\item Nekvalifikuoti darbuotojai. \\
  		Kaip naujokai sugeba susidoroti su i\v skilusiom problemom, \\ kiek \v zmoni\k u
  		pra\v sosi \k i darb\k a ir kiek patenk\k a, ar apklausose i\v sangrin\. ejamos
  		b\= u tiniausios detal\. es. Taip pat kaip  darbuotojai supranta dokumentacij\k a.

	\item Prasta produkto kokybe, nei\v sbaigtos produkto dalys.\\
		kas vyksta su darb\k u kokyb\k e did\. ejant darb\k u spaudimui, art\. ejant \\
		projekto terminui, steb\. eti komandos nari\k u spendimus.
  		
	\item Klaid\k u \k iv\. elimas, laiko praleidimas.\\
  		Kaip sudarin\. ejamas darbo grafikas, kaip darbuotojai \k ivertina tai k\k a jiem
  		reikia padaryti.
  		
	\item Produkto kokybe blogesn\. e,	programinis nesusiderinamumas. \\ 
  		Santykius tarp bendradarbi\k u, komunikavim\k a esant grup\. ese, bendro darbo 
  		naud\k a ir atskiro darbo naud\k a.
  		
	\item Produkto dokumentacija nepateikiama laiku.
  		Steb\. eti kaip ra\v omas taisomas dokumentas, kada pristabdomas dokumento ra\v symas
  		ir kada ra\v sant dokument\k a atsiranda problemos.

	
  		
\end{enumerate}

\newpage

\subsection{Rizikos valdymas}
\begin{enumerate}
	\item Komunikavimo tr\= ukumas . \\
 			Keisti komuniavimo b\v uda, truputi pastum\. eti u\v zsakova jud\. eti pirmyn.
  			
  	\item Produktas neatitinka reikalavim\k u . \\
		Spr\k esti, keisti reikalavimus su u\v sakovu, pastum\. eti komanda dirbti toliau.
		
	\item Darbo vietos neu\v ztikrinimas.\\
  			Padaryti greit\k a apklaus\k a kas napatinka darbo vietoje, i\v snagrin\. eti\\
  			apklaus\k a ir i\v staisyti neprivalumus, padidinti ma\v zus atlyginimus
  		
	\item Nekvalifikuoti darbuotojai. \\
			Per\v ziur\. eti projekto dokument\k a \k isitikinti, kad visi projekto aspektai
			yra i\v sai\v skinami.
			Jei dokumentacija gera tai atleisti labiausiai nekvalifikuotus, kurie projekt\k a labiausiai stabdo. \\
			Kitu atveju tiesiog patobulinti dokumentacij\k a
			.
  	
	\item Prasta produkto kokybe, nei\v sbaigtos produkto dalys.\\
			Nusisamdyti \v zmogu gerai i\v smananti nei\v sbaigt\k u produkto dali\k u 
			k\= urime, per\v ziur\. eti projekto dokument\k a \k isitikinti, 
			kad visi projekto aspektai yra i\v sai\v skinami. \\ 
			Susisiekti su u\v zsakovu, keisti produkt\k a.
		
	\item Klaid\k u \k iv\. elimas, laiko praleidimas.\\
  			Per\v zvelgti darb\k u planus, \k ivertinti grafikus.
  			Jei jie geri, tai atleisti daugiausiai klaid\k u privelen\v cius, projekt\k a
  			stabda\v cius darbuotojus, nusisamdyti naujus darbuotojus. Jei ne tada \k isakyti
  			sudaryti gera darb\k u plan\k a ir kartoti veiksm\k a nuo prad\v zi\k u.
  		
	\item Produkto kokybe blogesn\. e, programinis nesusiderinamumas. \\ 
  			Nusisamdyti papildomus darbuotojus gerai nusimananti\v cius ir komunikuojan\v cius
  			ir sudaryti \k ivesti \k i sudaryta grup\k e ar sudaryti nauj\k a
  		
	\item Produkto dokumentacija nepateikiama laiku.\\
  			Nusisamdyti daugiau \v zmoni\k u ruo\v sti dokumentacij\k a.


\end{enumerate}

\section{Konfiguracin\. e kontrol\. e}
Kekvieno produkto gyvavimo cikle atsiranda nei\v sviangiam\k u pasikeitim\k u.\\
Konfiguracin\. e kontrol\. e u\v zikrina, kad tie pasikeitimai b\= utu minimal\= us.
\v Siame skyriuje aptarsiu galimus pasikeitimai i\v s u\v zakovo pus\. es bei kaip juos kontroliuoti. \\
\\
U\v zsakovas gali nuspr\k esti
\begin{itemize}
	\item pakeisti produkto informacij\k a.
	\item kada ir kiek i\v spl\. esti, pagreitinti programos funkcionalum\k a.
	\item koki? servis\k u papildymo tr\= uksta.
\end{itemize}

Kekvin\k a u\v zsakovo sprendim\k a turime ivykdyti. Problema tame, kad reikia atsekti kada ir kur ivykdyti pakeitim\k a.\\
Problemos sprendimas ir b\= utu konfiguracin\. es kontrol\. es naudojimas.\\
Ji u\v ztikrina, kad informacija saugojama, kontroliuojama sistemos k\= urimo, keitimo ir testavimo metu ir taip pat ar atitinka vykdomus \v reikalavimus. \\
Taip pat u\v ztikrinti kad informacij\k a apie projekta laisvai pasiekiama.\\
Konfiguracin\k e kontrol\. es u\v zuduotys:
\begin{enumerate}
	\item Rasti kokios projekto dalys gali keistis.
	\item kontroliuoti vien\k a ar kelis rastus.
	\item lengvinti skirting\k u versij\k u aplikacij\k u konstrukcij\k a. 
	\item u\v ztikrinti kad projekto kokybe palaikoma, kai konfiguracija atnaujinama.
\end{enumerate}
\newpage
Konfiguracijos kontrole u\v ztikrina programos konfiguracijos element\k u identifikavim\k a, pasikeitimo kontrol\k e, versijos kontrol\k e, konfiguracijos audit\k a ir ataskaita tiek projekto k\= urimo metu, tiek palaikymo metu.\\\\


\large\textbf{Projekte kontrol\. e u\v ztikrinama}
\normalsize
\begin{enumerate}
	\item Programos dokumentacijai.
	\item Programos analiz\. es failai.
	\item Programos diegimo failam.
	\item Sistemos failam.
\end{enumerate}

\

\section{Versij\k u kontrol\. e}

Versij\k u kontrolei naudojamas "Github".\\
Ne tik programos kodui versijonuoti bet taip pat ir apie tai informuoti bendradarbius.\\

\large\textbf{Joje laikomi failai:}
\normalsize
\begin{enumerate}

	\item Vadyba ( organizacijos strukt\= ura, architekt\= uros informacija ).
	\item Modeliavmas ( analiz\. e, dizainas ).
	\item Konstrukcijos ( \v saltinio kodas, kompiliavimo instrukcijos )
	\item Testai ( skriptai, rezultatai, kokyb?s matavimai ).
	\item Dokumentai (projekto, dizaino dokumentai, naudojimo \v zinynas ).
	\item Projekto valdymas ( projekto trukm\. e, tvarkara\v stis, auditas ).
		
\end{enumerate}


\end{document}