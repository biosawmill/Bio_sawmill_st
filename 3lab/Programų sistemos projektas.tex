\documentclass[a4paper,12pt]{article}
\usepackage[utf8x]{inputenc}
\usepackage[T1]{fontenc}

%\usepackage[T2A]{fontenc} % jei yra kirilica
\usepackage[hmargin={30mm,15mm},vmargin={20mm,20mm},bindingoffset=0mm]{geometry}
\usepackage[onehalfspacing]{setspace}
\usepackage[colorlinks=true, linkcolor=blue, citecolor=blue, urlcolor=blue, unicode]{hyperref}

%\parindent=7mm
\renewcommand{\refname}{Literatūros sąrašas} % article
%\renewcommand{\bibname}{Literatūros sąrašas} % report
\renewcommand{\contentsname}{Turinys}
\usepackage[T1]{fontenc} 

% Lukas paketai
\usepackage[capposition=top]{floatrow}
\usepackage{lmodern,textcomp}
\usepackage{booktabs}% http://ctan.org/pkg/booktabs
\newcommand{\tabitem}{~~\llap{\textbullet}~~}
\usepackage{graphicx}
\usepackage{verbatim}
\usepackage{indentfirst}
\usepackage{setspace}
\usepackage{placeins}
\usepackage{booktabs}% http://ctan.org/pkg/booktabs
\usepackage{tabularx}% http://ctan.org/pkg/tabularx
\usepackage[parfill]{parskip}
\usepackage[unicode]{hyperref}
\usepackage{hyperref}
\usepackage{tocloft}
\usepackage{graphicx}\newcommand\AtPageUpperRight[1]{\AtPageUpperLeft{%
   \makebox[\paperwidth][r]{#1}}}
\usepackage[dotinlabels]{titletoc}
\usepackage[capposition=top]{floatrow}
\hypersetup{
    colorlinks,
    citecolor=black,
    filecolor=black,
    linkcolor=black,
    urlcolor=black
}
\usepackage{secdot}




\begin{document}
\graphicspath{./}

\renewcommand{\cftdot}{.}	
\renewcommand{\cftsecleader}{\cftdotfill{\cftdotsep}}

\thispagestyle{empty} % nerasomas psl. nr


\begin{center}
 VILNIAUS UNIVERSITETAS 
 
MATEMATIKOS IR INFORMATIKOS FAKULTETAS

MATEMATINĖS INFORMATIKOS KATEDRA

\vspace{4cm}

Projekto vadovas \ \ \textbf{Lukas Tutkus} \\
\textbf{Julius Daukšas} \\
\textbf{Dominykas Smaliukas} \\
\textbf{Robert Stankevič} \\

\vspace{0.2cm}

Bioinformatikos studijų programos grupė BioSawmill



\vspace{3cm}
\textbf{\Large Programų sistemos projektas}\\


\vfill

Vilnius \ \  2015
\end{center}



\clearpage

\tableofcontents
\clearpage


\section{Klasių diagrama}
\includegraphics[scale=0.5]{ClassDiagram}

\section{Duomenų bazės struktūra}
Duomenų bazės bus sugeneruota su įrankiu - Hibernate.\\
Hibernate įrankui bus nenaudojama tik "Optimizacijos" klasė.


\section{Sistemos architektūra}
\subsection{Serverio - mašinos, kurioje bus diegiama sistema - aplinka}
\includegraphics[scale=0.5]{architektura1}
\subsection{Aplikacijos architektūta}
\includegraphics[scale=0.5]{architektura2}
\subsection{Aplikacijos komponentai}
\includegraphics[scale=0.5]{architektura4}
\subsection{"Deployment" architektūra}
\includegraphics[scale=0.5]{architektura3}




\section{Sudėtingesnių modulių veikimo apibrėžimai}

\subsubsection{P.O. mokestis}
P.O. mokėjimą galima atlikti po prisijungimo. \\
Vartotojui paspaudus mokėjimo mygtuką, atsiranda mokėjimo langas, kuriame nurodoma:
\begin{itemize}
	\item Mokėjimo laikotarpis.
	\item Data laikotarpio pradžiai. 
\end{itemize} 

Vartotojas gali tiek užmokėti už optimizacijos naudojio pradžia tiek pratęsti esanti laikotarpį.
Pavyzdžiui, mokėjimo laikotarpiai yra mėnesis(5€) ir metai(45€). Galima nurodyti kelis mėnesius ar metus. \\
Būtinos sąlygos:
\begin{itemize}
	\item Data ne ankstesnė už šiandienos bei už paskutinę galiojimo dieną.
	\item Laikotarpį galima nurodyti ne mažesnį už mėnesį. Jei nurodoma 12 mėnesių tada mokoma už metus.
\end{itemize}

Vartotojui automatiškai parenkamas bankas(default), jis, žinoma, gali pakeisti į norimą.
Paspaudžiamas "Mokėti" mygtukas.
Jei duomenų validavimas gražino klaida, vartotojui pranešama, kur įvyko klaida.
Vartotojas nusiunčiamas į nurodyt banko svetainę, kurioje turi atlikti optimizacijos mokęstį.
Tinkamai atlikus pavedimą, vartotojos priklausomai nuo nurodytos paslaugos aktyvavimo datos ir pasirinkto laikotarpio ilgio gali naudotis pjovimo optimizacija.\\
Administratorius paskyros lange gali keisti paslaugos mokestį. Vartotojai, kurie buvo užsimokėję už paslaugą po mokesčio sumos pakeitimo, lieka nepaliesti pokyčių. Mėnesinis mokestis nurodomas naturaliaisiais skaičiais, formulėje gali būti tik 2 daugikliai: kooefecientas K, galintis būti realusis teigiamas skaičius, ir M - naturalusis skaičius (pagal idėją tai turėtų būti standartinis 12 mėnesių mokestis, bet administratorius gali nurodyti ir kitokį skaičių).
	
\subsection{ P.O. duomenų įvedimas }

Duomenys galima įvesti tiek rankiniu tiek failo ikėlimo budu.\\
Nežinantiem kaip užpildyti failą galima atsisiūsti pavizdinį failą\\
paspaudus mygtuką "Pavizdinio failo parsisiuntimas".\\
Paspaudus mygtuką "Istorija" galima peržvelgti praeitus optimizacijos \\
duomenis ir juos panaudoti įvedime.\\
Pabaigus įvedimą spaudžiamas "Optimizuoti mygtukas", atliekamas duomenų\\ validavimas. O tik po teisingo duomenų validavimo pradedama optimizacija. Mokestis apskaičiuojamas

\subsubsection{Rankinis įvedimas}

Panelių įvedime pateikiama:
\begin{itemize}
	\item dvi stanartinės panelės dydžių: 1200x2500 ir 1200x3050
	\item viena pasirenkamo dydžio standartinė panelė
	\item Mygtukas "Pridėti naują" pridedantis pasirenkamo dydžio panelę. 
\end{itemize} 
Visos panelės yra pasirenkamos, vartotojas, privalo pasirinkti bent vieną.

Detalių įvedime pateikiama:
\begin{itemize}
	\item Detalės kiekio įvedimas
	\item Detalės ilgio įvedimas
	\item Detalės pločio įvedimas
	\item Detalės pridėjimo mygtukas.
\end{itemize}

\subsubsection{Duomenų validavimas}
\textbf{Tikrinamas standartinių panelių validumas}:\\
	Tikrinama ar yra pasirinkta bent viena panelė. \\
	Toliau tikrinama ar pasirinktų panelių ilgai ir pločiai nuo 10mm iki 10000mm \\


\textbf{Tikrinamas detalių validumas}:\\
	Surandama mažiausias ilgis bei plotis iš duotų panelių .\\
	Tada tikrinama ar detalės ilgis ir plotis yra mažesnis ar lygus už rastus mažiausius ilgius, pločius.\\
	Tikrinama ar detalių bendaras kiekis neviršija 100000. \\

Jei validavimas teisingas vykdoma pjovimo optimizacija.


\subsubsection{Įvedimas iš failo}

Faile  kekviena eilutė turi tuėti 5 stulpelius:
\begin{enumerate}
	\item Panelės ilgis
	\item Panelės plotis
	\item Detalės kiekis
	\item Detalės ilgis
	\item Detalės plotis.
\end{enumerate}

Jeiu vartotojas nurodo panelės ilgį bei panelės plotį - 0, tada panelė praleidžiama ir skaitoma detalės duomenys. \\
Reikalaujama įvesti bent viena eilutė su panelės ilgiu ir pločiu nelygiu 0 ir bet bent viena detalė.

\subsubsection{Detalių .csv failo validavimas}
\textbf{Tikrinamas detalių validumas}:\\
	Jei failas tuščias grįžtama į įvestį, kur pranešama kad failas tuščias. \\
	Jei faile nėra kablelių pranešama kad failas blogo formato(nėra kablelių). \\
	Jei failo eilutėje nėra lygiai 5 reikšmių, pranešama kad blogo formato(neteisingas reikšmių sk.) pvz: 0, 0, 1, 10 ( 4 reikšmės).
	
	
	Panelių ir detalių dydžiai bei detalių kiekis turi atitikti duomenų validacijos standartus. 
Tinkamos eilutės pvz.: 150, 150, 2, 75, 80. \\
	Reiškia kad įvesta standartinė panelė 150 (mm) ilgio ir pločio bei \\
	įvestos 2 detalės su 75x80(mm) matmenimis



\subsection{"Istorijos" mygtukas bei funkcija}

\subsubsection{Pjūvio plano informacijos išsaugojimas}
Pasirenkamas reikalinga rezultatų ruošinys. \\
Pateikiamas plano pavadinimo įvedimas, kuriuo bus saugomi duomenys. \\
Paspaudžiamas mygtukas saugoti. \\
Plano pavadinimas įkeliamas į mygtuko "Istorija" sąraše jei dar nėra 10 ruošinių.\\
Kitu atveju atsiranda pranešimas: \\
"Jūsų istorijoje jau yra saugoma 10 įrašų. Uždarykite įš langelį ir istorijoje pašalinkite vieną iš įrašų, kad galėtumėte išsaugoti šitą planą". \\
Teisingu atveju gauti duomenys bei ruošiniai išsaugomi duomenų bazėje.

\subsubsection{"Istorijos" peržiūra}
Prisijungus bei paspaudus mygtuką "Istorija" atsiranda Istorijos langas.
Istorijos saraše naujiausi planai išsaugoti-aukščiausiai, seniausi išsaugoti žemiausiai.
Prie plano pavadinimo bei datos pateikiami 5 to plano panaudojimo mygtukai.
\begin{enumerate}
	\item Paspaudus mygtuką "Naudoti duomenis" nukeliama į duomenų skilty.
		Ten jau suvesti pasirinkto plano duomenys.
	\item Paspaudus palno peržiūrėjimo ikoną pateikiami išsaugoti rezultatai:
		\begin{itemize}
			\item detalių išdėstymas
			\item sunaudotas panelių kiekis
			\item bendras detalių plotas
			\item panelių likutis
			\item bendras pjovimo ilgis.
		\end{itemize}
	\item Paspaudus atsisiuntimo formos mygtuką atsiunčiami minėti duomenys ".pdf" formatu.
	\item Paspaudus šiukšliadežės formos mygtuką ištrinamas išsaugotas planas.
	\item Paspaudus "floppy disk" formos mygtuką išsaugomas pakeistas plano pavadinimas.
	\item Paspaudus "Peržiūrėti pradinius duomenis" formos mygtuką vartotojui pateikiami pradiniai duomenys, naudoti konkrečiame projekte.
\end{enumerate}
\clearpage

\section{Interfeisai}

\subsection{Pastabos}
Interfeisuose matomas geltono kontūro langas yra vadinamasis "pop-up" langas. Jo veikimo pavyzdys galėtų būti toks: paprastas vartotojas prisijungia prie savo paskyros lango, jame mato mygtuką "Mokėti", vartotojui paspaudus šį mygtuką paskyros lango turinys yra paslepiamas ir atsiranda geltono kontūro mokėjimo dialogas. Šiame dialoge esantis "X" mygtukas raudoname fone leidžia vartotojui grįžti prie ir vėl matyti paslėptąjį turinį (su visa jame prieš tai buvusia informacija) ir, žinoma, uždaro geltoname kontūre buvusį dialogą, užmirštant visą jame buvi įvestą turinį.\\
Atsijungimo mygtukas matomas tik prisijungus. O jį paspaudus, vartotojas nukreipiamas į namų puslapio langą.
Optimizacijoje atsisiunčiamų rezultatų .pdf formatu kalba priklausys nuo nustatytos svetainėje, tarkim paspaudus anglų kalbą ir paspaudus mygtuką atsisiūsti gaunamas anglų kalbos .pdf failas.


\subsection{Namų puslapis}
\floatfoot{1 interfeisas. Namų puslapis (vartotojas prisijungęs)}
\begin{figure}[!tph]
\hspace{-2cm}
\centering
\includegraphics[scale=0.45]{interfeisai/pagrindinis}
\label{fig:verticalcell}
\end{figure}

\subsection{Vartotojo paskyros puslapis}
\floatfoot{2 interfeisas. Vartotojo paskyros puslapis}
\begin{figure}[!tph]
\hspace{-2cm}
\centering
\includegraphics[scale=0.45]{interfeisai/paskyrosPuslapisNeprisijungta}
\label{fig:verticalcell}
\end{figure}

\floatfoot{I1. 3 interfeisas. Vartotojo paskyros puslapis - registracija}
\begin{figure}[!tph]
\hspace{-2cm}
\centering
\includegraphics[scale=0.45]{interfeisai/paskyrosPuslapisRegistracija}
\label{fig:verticalcell}
\end{figure}

\floatfoot{4 interfeisas. Vartotojo paskyros puslapis - registracija (su klaida)}
\begin{figure}[!tph]
\hspace{-2cm}
\centering
\includegraphics[scale=0.45]{interfeisai/paskyrosPuslapisRegistracijaSuKlaida}
\label{fig:verticalcell}
\end{figure}

\floatfoot{5 interfeisas. Vartotojo paskyros puslapis - sėkminga registracija}
\begin{figure}[!tph]
\hspace{-2cm}
\centering
\includegraphics[scale=0.45]{interfeisai/paskyrosPuslapisRegistracijaSekminga}
\label{fig:verticalcell}
\end{figure}
 
\floatfoot{I2. 6 interfeisas. Vartotojo paskyros puslapis - prisijungimas. Sėkmingai prisijungus, vartotojas nukreipiamas į 10 arba 11 stiliaus interfeisą, priklauso nuo to ar jis yra užsimokėjęs už paslaugas.}
\begin{figure}[!tph]
\hspace{-2cm}
\centering
\includegraphics[scale=0.45]{interfeisai/paskyrosPuslapisPrisijungimas}
\label{fig:verticalcell}
\end{figure}

\floatfoot{7 interfeisas. Vartotojo paskyros puslapis - prisijungimas (su klaida)}
\begin{figure}[!tph]
\hspace{-2cm}
\centering
\includegraphics[scale=0.45]{interfeisai/paskyrosPuslapisPrisijungimasSuKlaida}
\label{fig:verticalcell}
\end{figure}

\floatfoot{I3. 8 interfeisas. Vartotojo paskyros puslapis - pamirštas slaptažodis. Vartotojui paspaudus "Priminti slaptažodį" jis nukreipiamas į 9 interfeisą.}
\begin{figure}[!tph]
\hspace{-2cm}
\centering
\includegraphics[scale=0.45]{interfeisai/paskyrosPuslapisPamirstasSlaptazodis}
\label{fig:verticalcell}
\end{figure}

\floatfoot{9 interfeisas. Vartotojo paskyros puslapis - pamirštas slaptažodis (po mygtuko "Priminti slaptažodį" paspaudimo). Vartotojui paspaudus "X" jis yra gražinamas į 2 interfeisą. Vartotojui paspaudus "Bandyti dar kartą" jis yra gražinamas į 8 interfeisą.}
\begin{figure}[!tph]
\hspace{-2cm}
\centering
\includegraphics[scale=0.45]{interfeisai/paskyrosPuslapisPamirstasSlaptazodis2}
\label{fig:verticalcell}
\end{figure}


\floatfoot{10 interfeisas. Vartotojo paskyros puslapis - prisijungus (neapmokėta už paslaugas). Paspaudus mygtuką "Mokėti" vartotojas mato 12 interfeisą (dialogą).}
\begin{figure}[!tph]
\hspace{-2cm}
\centering
\includegraphics[scale=0.45]{interfeisai/paskyrosPuslapisVartotojasNeapmoketas}
\label{fig:verticalcell}
\end{figure}

\floatfoot{11 interfeisas. Vartotojo paskyros puslapis - prisijungus (apmokėta už paslaugas ir su klaida)}
\begin{figure}[!tph]
\hspace{-2cm}
\centering
\includegraphics[scale=0.45]{interfeisai/paskyrosPuslapisVartotojasApmoketas}
\label{fig:verticalcell}
\end{figure}

\floatfoot{12 interfeisas. Vartotojo paskyros puslapis - prisijungus (mokėjimas). Atidarytame lange datos laike turi būti iš anksto įrašyta data atitinkanti 4.0.1. Įvykus sėkmingam apmokėjimui, mokėjimo langas dingsta, vartotojui gražinamas paskyros lango turinys, tačiau turinys yra nebe 10, o 11 interfeiso dizaino tipo.}
\begin{figure}[!tph]
\hspace{-2cm}
\centering
\includegraphics[scale=0.45]{interfeisai/paskyrosPuslapisVartotojasMokejimas}
\label{fig:verticalcell}
\end{figure}

\floatfoot{13 interfeisas. Vartotojo paskyros puslapis - prisijungus (mokėjimas su klaida)}
\begin{figure}[!tph]
\hspace{-2cm}
\centering
\includegraphics[scale=0.45]{interfeisai/paskyrosPuslapisVartotojasMokejimasSuKlaida}
\label{fig:verticalcell}
\end{figure}


\floatfoot{14 interfeisas. Vartotojo paskyros puslapis - administratoriaus matomas langas. Administratorius gali ieškoti vartotojų pagal vardą arba el. paštą (rezultatai rikiuojami pagal vartotojo vardą abėcėlės tvarka). Į paieškos langą įvedus specialųjį simbolį "*" bus pateikiamas visų vartotojų sąrašas. Tokiu atveju, kai gražinamas vartotojų sąrašas turi daugiau negu 10 vartotojų, turi būti pateikiami pirmieji 10 sąraše esančių vartotojų. Likę vartotojai pateikiami administratoriui "scrolinant" per sąrašą, kaip yra parodyta 23.1 interfeise. Administratorius gali keisti mokestį už paslaugas.}
\begin{figure}[!tph]
\hspace{-2cm}
\centering
\includegraphics[scale=0.45]{interfeisai/paskyrosPuslapisAdministratorius2}
\label{fig:verticalcell}
\end{figure}

\clearpage

\floatfoot{15 interfeisas. Vartotojo paskyros puslapis - administratoriaus matomas langas (su klaida). "Floppy disc" ikonos paspaudimas išsaugo pakeitimus padarytus konkrečiai vartotojo paskyrai, o šiukšliadėžės ikonos paspaudimas leidžia ištrinti vartotojo paskyrą prieš tai nukrepiant vartotoją į 15.1 dialogą.}
\begin{figure}[!tph]
\hspace{-2cm}
\centering
\includegraphics[scale=0.45]{interfeisai/paskyrosPuslapisAdministratoriusSuKlaida2}
\label{fig:verticalcell}
\end{figure}

\floatfoot{15.1 interfeisas. Vartotojo paskyros puslapis - administratorius nori ištrinti vartotoją. Nesvarbu, kurį mygtuką paspaus administratorius, jis bus gražinamas į 15 interfeisą (nebent įvyktų nenumatyta klaida serveryje).}
\begin{figure}[!tph]
\hspace{-2cm}
\centering
\includegraphics[scale=0.45]{interfeisai/vartotojoPaskyraVartotojoIstrynimoPatvirtinimas}
\label{fig:verticalcell}
\end{figure}



\clearpage

\subsection{Optimizavimo puslapis}
\floatfoot{16 interfeisas. Optimizavimo puslapis - neprisijungus.}
\begin{figure}[!tph]
\hspace{-2cm}
\centering
\includegraphics[scale=0.45]{interfeisai/optimizavimoPuslapisNeprisijungus}
\label{fig:verticalcell}
\end{figure}

\floatfoot{17 interfeisas. Optimizavimo puslapis - prisijungus. Vartotojui paspaudus "Pridėti naują" atsiradna nauji duomenų įvedimo laukeliai atitinkamai  prie detalių arbą panelių. \\
"Importuoti iš failo (csv)" mygtukas atidaro standartinį operacinės sistemos failo pasirinkimo langą.\\
Failas turi atitikti sudetingesnių modelių aprašyme nustatytą standartą.
 "Optimizuoti pjūvio planą" mygtukas atodaro 19 interfeisą (dialogą), o "Istorijos" mygtukas 23-ią interfeisą (dialogą).}
\begin{figure}[!tph]
\hspace{-2cm}
\centering
\includegraphics[scale=0.45]{interfeisai/optimizavimoPuslapisPrisijungus}
\label{fig:verticalcell}
\end{figure}

\floatfoot{18 interfeisas. Optimizavimo puslapis su klaidos pranešimu}
\begin{figure}[!tph]
\hspace{-2cm}
\centering
\includegraphics[scale=0.45]{interfeisai/optimizavimoPuslapisPrisijungusSuKlaida}
\label{fig:verticalcell}
\end{figure}

\floatfoot{18.1 interfeisas. Optimizavimo puslapis - optimizavimo procesas. Paspaudus "X" grįžtama į optimizavimo interfeisą 17. Pasibaigus optimizavimui vartotojas turi būti automatiškai nukreiptas į 19 interfeisą. }
\begin{figure}[!tph]
\hspace{-2cm}
\centering
\includegraphics[scale=0.45]{interfeisai/optimizuojama}
\label{fig:verticalcell}
\end{figure}
\clearpage

\floatfoot{19 interfeisas. Optimizavimo puslapis - po optimizavimo proceso. "Peržiūrėti planą" mygtukas atidaro 20 interfeisą (dialogą). Paspaudus "X" mygtuką visi neišsaugoti optimizacijos gauti rezultatai bus pamiršti.}
\begin{figure}[!tph]
\hspace{-2cm}
\centering
\includegraphics[scale=0.45]{interfeisai/optimizavimoPuslapisOptimizuotiPlanai}
\label{fig:verticalcell}
\end{figure}
\clearpage

\floatfoot{20 interfeisas. Optimizavimo puslapis - po optimizavimo, pasirinkus vieno iš planų peržiūrą (iš arčiau). Mygtukas "Parsisiųsti" atidaro standartinį operacinės sistemos failo išsaugojimo langą pjūvio plano ruošiniui (.pdf) atsisiuntimui. Išsaugojimo mygtukas atidaro 21 interfeisą (dialogą), o  "Istorijos" mygtukas 23 interfeisą (dialogą).}

\begin{figure}[!tph]
\hspace{-2cm}
\centering
\includegraphics[scale=1]{interfeisai/optimizavimoPuslapisPrisijungusPasirinktoPerziura2}
\label{fig:horizontalcell}
\end{figure}


\floatfoot{21 interfeisas. Optimizavimo puslapis - pasirinkto plano išsaugojimas. Paspaudus mygtuką "Išsaugoti" šis dialogas yra uždaromas ir grįžtama į 19 dialogą, paspaudus "X" mygtuką nevykdomas išsaugojimas ir grįžtama į 19 dialogą.}
\begin{figure}[!tph]
\hspace{-2cm}
\centering
\includegraphics[scale=0.45]{interfeisai/optimizavimoPuslapisPrisijungusPasirinktoPlanoIsaugojimas}
\label{fig:verticalcell}
\end{figure}

\floatfoot{22 interfeisas. Optimizavimo puslapis - pasirinkto plano išsaugojimas (su klaida)}
\begin{figure}[!tph]
\hspace{-2cm}
\centering
\includegraphics[scale=0.45]{interfeisai/optimizavimoPuslapisPrisijungusPasirinktoPlanoIsaugojimasSuKlaida}
\label{fig:verticalcell}
\end{figure}


\begin{comment}
%------- UZKOMENTUOTA ------
\floatfoot{23 interfeisas. Optimizavimo puslapis - istorijos peržiūra. "Floppy disc" ikonos mygtukas išsaugo konkretaus įrašo pakeitimus. Šiukšlinės ikonos paspaudimas ištrina konkretų įrašą, o atsisiuntimo ikona atidaro standartinį operacinės sistemos duomenų išsaugojimo langą. "Peržiūrėti planą" mygtukas atidato 25 interfeisą (dialogą).}
%------- UZKOMENTUOTA PABAIGA------
\end{comment}
\floatfoot{23 interfeisas. Optimizavimo puslapis - istorijos peržiūra. "Floppy disc" ikonos mygtukas išsaugo konkretaus įrašo pakeitimus, šiukšlinės ikonos paspaudimas ištrina konkretų įrašą, o atsisiuntimo ikona atidaro standartinį operacinės sistemos duomenų išsaugojimo langą. Dokumento peržiūros ikonos paspaudimas atidaro 25 interfeisą, o "Peržiūrėti pradinius duomenis" mygtukas atidato 23.1 interfeisą (dialogą).}
\begin{figure}[!tph]
\hspace{-2cm}
\centering
\includegraphics[scale=0.45]{interfeisai/optimizavimoPuslapisPrisijungusIstorija2}
\label{fig:verticalcell}
\end{figure}

\floatfoot{23.1 interfeisas. Optimizavimo puslapis - išsaugoto projekto pradinių duomenų peržiūra. Esant didesniam panelių kiekui, jos būtų atvaizduojamos dėliojant jas šalia (dešiniau) jau esančių pavyzdyje. Esant didesniam negu šeši detalių kiekiui, detalių atvaizdavimo stačiakampuje turi atsirasti "sidebar", leidžiantis peržiūrėti kitas (pavyzdyje nematomas) detales.}
\begin{figure}[!tph]
\hspace{-2cm}
\centering
\includegraphics[scale=0.45]{interfeisai/optimizavimoPuslapisPradiniuDuomenuPerziura}
\label{fig:verticalcell}
\end{figure}


\floatfoot{24 interfeisas. Optimizavimo puslapis - istorijos redagavimas (su klaida)}
\begin{figure}[!tph]
\hspace{-2cm}
\centering
\includegraphics[scale=0.45]{interfeisai/optimizavimoPuslapisPrisijungusIstorijaSuKlaida2}
\label{fig:verticalcell}
\end{figure}

\clearpage


\floatfoot{25 interfeisas. Optimizavimo puslapis - išsaugoto plano istorijoje peržiūra (iš arčiau). "X" mygtukas gražina vartotoją į 23 dialogą, o parsisiuntimo - atidaro standartinį operacinės sistemos failo išsaugojimo langą. Lange matomas "x8" nurodo kokiam kiekiui planlių reikės taikyti vaizduojamą pjųvio planą. Plano apačioje esančios dvi ikonos (geltona ir juoda) leidžia perjungti pjūvio plano vaizdą likusiam panelių kiekui. Vartotojo atsisiųstame pjūvio ruošinyje (pagal šį pavyzdį) vartotojas matytų šious du skirtingus planus einančius vienas po kito.} 
\begin{figure}[!tph]
\hspace{-2cm}
\centering
\includegraphics[scale=1]{interfeisai/optimizavimoPuslapisPrisijungusIsaugotasPlanas}
\label{fig:verticalcell}
\end{figure}

\clearpage

\subsection{Susisiekimo su mumis puslapis}


\floatfoot{26 interfeisas. Susisiekimo su mumis puslapis}
\begin{figure}[!tph]
\hspace{-2cm}
\centering
\includegraphics[scale=0.45]{interfeisai/susisiekimas}
\label{fig:verticalcell}
\end{figure}

\floatfoot{27 interfeisas. Susisiekimo su mumis puslapis (su klaida)}
\begin{figure}[!tph]
\hspace{-2cm}
\includegraphics[scale=0.45]{interfeisai/susisiekimasSuKlaida}
\label{fig:verticalcell}
\end{figure}

\subsection{ Nenumatytos klaidos atvejis}
\floatfoot{28 interfeisas. Nenumatyta klaida, pateikiamas hyperlink'as į Susisiekite su mumis langą.}
\begin{figure}[!tph]
\hspace{-2cm}
\centering
\includegraphics[scale=0.45]{interfeisai/bendraKlaida}
\label{fig:verticalcell}
\end{figure}

\clearpage

% ---------------------------------- UZKOMENTUOTA, NEISTRINTI  ------------------------------
\begin{comment}
\subsection{Namų puslapis (vartotojas prisijungęs)}
\subsection{Namų puslapis}


\hspace{-2cm}
\includegraphics[scale=0.5]{interfeisai/pagrindinis}

\subsection{Vartotojo paskyros puslapis - vartotojas neprisijungęs}
\hspace{-2cm}
\includegraphics[scale=0.5]{interfeisai/paskyrosPuslapisNeprisijungta}

\subsection{Vartotojo paskyros puslapis - registracija}
\hspace{-2cm}
\includegraphics[scale=0.5]{interfeisai/paskyrosPuslapisRegistracija}

\subsection{Vartotojo paskyros puslapis - registracija (su klaida)}
\hspace{-2cm}
\includegraphics[scale=0.5]{interfeisai/paskyrosPuslapisRegistracijaSuKlaida}

\subsection{Vartotojo paskyros puslapis - sėkminga registracija}
\hspace{-2cm}
\includegraphics[scale=0.5]{interfeisai/paskyrosPuslapisRegistracijaSekminga}

\subsection{Vartotojo paskyros puslapis - prisijungimas}
\hspace{-2cm}
\includegraphics[scale=0.5]{interfeisai/paskyrosPuslapisPrisijungimas}

\subsection{Vartotojo paskyros puslapis - prisijungimas (su klaida)}
\hspace{-2cm}
\includegraphics[scale=0.5]{interfeisai/paskyrosPuslapisPrisijungimasSuKlaida}

\subsection{Vartotojo paskyros puslapis - pamirštas slaptažodis}
\hspace{-2cm}
\includegraphics[scale=0.5]{interfeisai/paskyrosPuslapisPamirstasSlaptazodis}

\subsection{Vartotojo paskyros puslapis - pamirštas slaptažodis 2}
\hspace{-2cm}
\includegraphics[scale=0.5]{interfeisai/paskyrosPuslapisPamirstasSlaptazodis2}

\subsection{Vartotojo paskyros puslapis - prisijungus (neapmokėta už paslaugas)}
\hspace{-2cm}
\includegraphics[scale=0.5]{interfeisai/paskyrosPuslapisVartotojasNeapmoketas}

\subsection{Vartotojo paskyros puslapis - prisijungus (apmokėta už paslaugas)}
\hspace{-2cm}
\includegraphics[scale=0.5]{interfeisai/paskyrosPuslapisVartotojasApmoketas}

\subsection{Vartotojo paskyros puslapis - prisijungus - mokėjimas}
\hspace{-2cm}
\includegraphics[scale=0.5]{interfeisai/paskyrosPuslapisVartotojasMokejimas}

\subsection{Vartotojo paskyros puslapis - prisijungus - mokėjimas (su klaida)}
\hspace{-2cm}
\includegraphics[scale=0.5]{interfeisai/paskyrosPuslapisVartotojasMokejimasSuKlaida}

\subsection{Vartotojo paskyros puslapis - administratorius}
\hspace{-2cm}
\includegraphics[scale=0.5]{interfeisai/paskyrosPuslapisAdministratorius}

\subsection{Vartotojo paskyros puslapis - administratorius (su klaida)}
\hspace{-2cm}
\includegraphics[scale=0.5]{interfeisai/paskyrosPuslapisAdministratoriusSuKlaida}

\subsection{Optimizavimo puslapis - neprisijungus}
\hspace{-2cm}
\includegraphics[scale=0.5]{interfeisai/optimizavimoPuslapisNeprisijungus}

\subsection{Optimizavimo puslapis - prisijungus}
\hspace{-2cm}
\includegraphics[scale=0.5]{interfeisai/optimizavimoPuslapisPrisijungus}

\subsection{Optimizavimo puslapis - prisijungus (su klaida)}
\hspace{-2cm}
\includegraphics[scale=0.5]{interfeisai/optimizavimoPuslapisPrisijungusSuKlaida}

\subsection{Optimizavimo puslapis - po optimizavimo proceso}
\hspace{-2cm}
\includegraphics[scale=0.5]{interfeisai/optimizavimoPuslapisOptimizuotiPlanai}

\subsection{Optimizavimo puslapis - po optimizavimo, pasirinkus vieno iš planų peržiūrą}
\hspace{-2cm}
\includegraphics[scale=0.5]{interfeisai/optimizavimoPuslapisPrisijungusPasirinktoPerziura}

\subsection{Optimizavimo puslapis - po optimizavimo, pasirinkus vieno iš planų peržiūrą (iš arčiau)}
\hspace{-2cm}
\includegraphics[scale=1.2]{interfeisai/optimizavimoPuslapisPrisijungusPasirinktoPerziura2}

\subsection{Optimizavimo puslapis - pasirinkto plano išsaugojimas}
\hspace{-2cm}
\includegraphics[scale=0.5]{interfeisai/optimizavimoPuslapisPrisijungusPasirinktoPlanoIsaugojimas}

\subsection{Optimizavimo puslapis - pasirinkto plano išsaugojimas (su klaida)}
\hspace{-2cm}
\includegraphics[scale=0.5]{interfeisai/optimizavimoPuslapisPrisijungusPasirinktoPlanoIsaugojimasSuKlaida}

\subsection{Optimizavimo puslapis - istorijos peržiūra}
\hspace{-2cm}
\includegraphics[scale=0.5]{interfeisai/optimizavimoPuslapisPrisijungusIstorija}

\subsection{Optimizavimo puslapis - istorijos redagavimas (su klaida)}
\hspace{-2cm}
\includegraphics[scale=0.5]{interfeisai/optimizavimoPuslapisPrisijungusIstorijaSuKlaida}

\subsection{Optimizavimo puslapis - išsaugoto plano istorijoje peržiūra (iš arčiau)}
\hspace{-2cm}
\includegraphics[scale=1.2]{interfeisai/optimizavimoPuslapisPrisijungusIsaugotasPlanas}

\subsection{Susisiekimo su mumis puslapis}
\hspace{-2cm}
\includegraphics[scale=0.5]{interfeisai/susisiekimas}

\subsection{Susisiekimo su mumis puslapis (su klaida)}
\hspace{-2cm}
\includegraphics[scale=0.5]{interfeisai/susisiekimasSuKlaida}

\end{comment}
% ---------------------------------- ATKOMENTUOTA ------------------------------

\section{Interfeisų atsekamumų lentelė}

\begin{frame}
\centering
\begin{tabular}{|c|c|}
\hline
\textbf{Interfeisai} 			& \textbf{Naudojimosi atvejai}                                                    \\ \hline
I3, I4, I5		      			& 1. Registracija                                                                     \\ \hline
I6, I7, I16						& 2. Prisiungimas prie sistemos 	\\ \hline
Visur							& 3. Atsijungimas nuo sistemos	\\ \hline
I8, I9 							& 4. Slaptažodžio atgavimas		\\ \hline	
I10, I11							& 5. Pakeisti paskyros duomenis	\\ \hline
I11, I12, I13					& 6. Mokestis už optimizacija		\\ \hline
I14, I15, I15.1					& 7. Paskyru trinimas				\\ \hline
I17, I18							& 8. Duomenų įvedimas   			\\ \hline
I18.1							& 9. Pjūvio plano optimizavimas	\\ \hline
I19								& 10. Pjūvio plano pasirinkimas	\\ \hline
I20, I23, I24, I25				& 11. Atspausdinti pjūvio planą	\\ \hline
I20, I21, I22, I23, I24			& 12. Išsaugoti pjūvio planą                                                                              \\ \hline

\end{tabular}
\end{frame}
\\
\vspace{0.1cm}
I1, I2, ... ,I27 - Žymimi interfeisų numeriai.\\
Visur-žymima kad visuose interfeisuose yra atitinkama sąlyga.

\clearpage

\section{Reikalavimų atsekamumų lentelė}
\begin{frame}
\centering
\begin{tabular}{|c|c|}
\hline
\textbf{Reikalavimai} & \textbf{Interfeisai}                                                           \\ \hline
F1                    & I2, I3, I4, I5			\\ \hline
F2                    & I6, I7, I16        	 	\\ \hline
F3                    & VISI  					\\ \hline
F4                    & I8, I9 					\\ \hline
F5                    & I10, I11, I14, I15, I15.1\\ \hline
F6                    & I10, I11, I12, I13		\\ \hline
F7                    & I17, I18					\\ \hline
F8                    & I17, I18					\\ \hline
F9                    & I18.1\\ \hline
F10                   & I19, I20                                                                         \\ \hline
F11                   & I19                                                                             \\ \hline
F12                   & I20, I23, I24, I25                                                                     \\ \hline
F13                   & I20, I21, I22			\\ \hline
F14                   & I17, I18, I19, I20, I23, I23.1 I24			\\ \hline
F15                   & VISI                                                                           \\ \hline
F16                   & \begin{tabular}[c]{@{}c@{}}I4, I7, I11, I13, I15,\\ I16, I18, I22, I24, I27\end{tabular} \\ \hline
\end{tabular}
\end{frame}\\
I1, I2, ... ,I27 - Žymimi interfeisų numeriai.\\
VISI-žymima kad visuose interfeisuose yra atitinkamas reikalavimas.
\end{document}
