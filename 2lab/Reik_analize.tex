\documentclass[a4paper,12pt]{article}
\usepackage[utf8x]{inputenc}
\usepackage[T1]{fontenc}

%\usepackage[T2A]{fontenc} % jei yra kirilica
\usepackage[hmargin={30mm,15mm},vmargin={20mm,20mm},bindingoffset=0mm]{geometry}
\usepackage[onehalfspacing]{setspace}
\usepackage[colorlinks=true, linkcolor=blue, citecolor=blue, urlcolor=blue, unicode]{hyperref}

%\parindent=7mm
\renewcommand{\refname}{Literatūros sąrašas} % article
%\renewcommand{\bibname}{Literatūros sąrašas} % report
\renewcommand{\contentsname}{Turinys}
\usepackage[T1]{fontenc} 

% Lukas paketai
\usepackage{booktabs}% http://ctan.org/pkg/booktabs
\newcommand{\tabitem}{~~\llap{\textbullet}~~}
\usepackage{graphicx}
\usepackage{indentfirst}
\usepackage{setspace}
\usepackage{placeins}
\usepackage{booktabs}% http://ctan.org/pkg/booktabs
\usepackage{tabularx}% http://ctan.org/pkg/tabularx
\usepackage[parfill]{parskip}
\usepackage[unicode]{hyperref}
\usepackage{hyperref}
\usepackage{tocloft}
\usepackage{graphicx}
\newcommand\AtPageUpperRight[1]{\AtPageUpperLeft{%
   \makebox[\paperwidth][r]{#1}}}
\usepackage[dotinlabels]{titletoc}
\usepackage[capposition=top]{floatrow}
\hypersetup{
    colorlinks,
    citecolor=black,
    filecolor=black,
    linkcolor=black,
    urlcolor=black
}
\usepackage{secdot}




\begin{document}
\graphicspath{ {/} }

\renewcommand{\cftdot}{.}	
\renewcommand{\cftsecleader}{\cftdotfill{\cftdotsep}}

\thispagestyle{empty} % nerasomas psl. nr


\begin{center}
 VILNIAUS UNIVERSITETAS 
 
MATEMATIKOS IR INFORMATIKOS FAKULTETAS

MATEMATINĖS INFORMATIKOS KATEDRA

\vspace{4cm}

Projekto vadovas \ \ \textbf{Lukas Tutkus} \\
\textbf{Julius Daukšas} \\
\textbf{Dominykas Smaliukas} \\
\textbf{Robert Stankevič} \\

\vspace{0.2cm}

Bioinformatikos studijų programos grupė BioSawmill



\vspace{3cm}
\textbf{\Large Reikalavimų specifikacija}\\

\vfill

Vilnius \ \  2015
\end{center}



\clearpage

\tableofcontents
\clearpage
\section{Funkciniai ir nefunkciniai reikalavimai}
\begin{frame}
\centering
\hspace{-2.5cm}
\label{my-label}
\begin{tabular}{|l|l|l|}
\hline
\textbf{ID}	& \textbf{Reikalavimai}							& \textbf{Grupės}  \\ \hline

F1	& 	\begin{tabular}[c]{@{}l@{}}
			Registracijos forma su 4 laukais: \\
			slapyvardžio, pašto, slaptažodžio, slaptažodžio patvirtinimo
		\end{tabular}										& 1	     		\\ \hline

F2	& 	\begin{tabular}[c]{@{}l@{}}
			Prisijungimo prie sistemos forma su 2 laukais: \\
			slapyvardžio ar pašto bei slaptažodžio.  		
		\end{tabular}										& 1, 3			\\ \hline
	
F3	& 	Atsijungimo nuo sistemos mygtukas					& 1, 3			\\ \hline


F4	& 	\begin{tabular}[c]{@{}l@{}}
			Slaptažodžio atgavimas mygtukas. \\
			Paspaudaus atsidaro forma su 	\\
			prisijungimo vardo lauko ivedimu.
		\end{tabular}							 			& 1, 3			\\ \hline

F5	& 	\begin{tabular}[c]{@{}l@{}}
			Paskyros parametrų keitimo forma su 3 laukais: \\
			prisijungimo vardo, pašto, slaptažodžio.					
		\end{tabular}										& 1, 3			\\ \hline 
		
F6	& 	\begin{tabular}[c]{@{}l@{}}
			Vartotojo paskyroje aktyvacijos	mygtukas, \\
			aktyvuotas vartotojas
			gali naudotis pjovimo optimizacija.	
		\end{tabular}										& 1				\\ \hline
		
		
F7	& 	\begin{tabular}[c]{@{}l@{}}
			Administratoriaus paskyroje atsirandantis \\
			mygtukas paspaudus tam tikrą vartotoją, \\
			naudojamas tos paskyros ištrinimui
		\end{tabular}										& 3				\\ \hline
		
F8	& 	\begin{tabular}[c]{@{}l@{}}
			Pjovimo optimizacijos duomenų laukai: \\
			standartinių panelių matmenys bei kiekiai \\
			detalių matmenys bei kiekiai. ".csv" failo įkėlimas. \\
		\end{tabular}										& 1, 3			\\ \hline
	
F9	& 	\begin{tabular}[c]{@{}l@{}}
			Pjūvio plano optimizavimas           	   			
		\end{tabular}										& 2				\\ \hline
		
F10	& 	\begin{tabular}[c]{@{}l@{}}
			Optimizuotų pjūvio planų informacijos pateikimas		
		\end{tabular}										& 2     			\\ \hline

F11	& 	\begin{tabular}[c]{@{}l@{}}
			Optimizuotų pjūvio planų variantų pasirinkimas		
		\end{tabular}										& 1, 3			\\ \hline

F12 & 	Pjūvio plano ruošinio ".pdf" formatu
		atsisiuntimas										& 1, 3			\\ \hline

F13 	& 	\begin{tabular}[c]{@{}l@{}}
			Pjūvio plano rezultatų išsaugojimas( plano pavadinimas,\\
			 data, panelių kiekis, bendras jų plotas, likuts, \\
			 detalių išdėstymas,	bendras pjovimo ilgis. )\\
			Taip pat pjūvio plano įvestų duomenų išsaugojimas. \\
		\end{tabular}										& 1, 3			\\ \hline

F14	& 	\begin{tabular}[c]{@{}l@{}}
			Kalbos pasirinkimas(LT, EN), su interfeiso \\
			pakeitimų jį paspaudus, bei išsaugojimu.
		\end{tabular}
									& 1, 3			\\ \hline
%% Prideti išsaugojima duomenų kita kart atsidarius.
F15 	& 	\begin{tabular}[c]{@{}l@{}}
			Mygtuko 	"Istorija" sukūrimas, kurį paspaudęs\\
			prisijungęs vartotojas mato iki 10 išsaugotų \\
			planų. Pasirinkus planą galima panaudoti įvedimo \\
			informaciją, peržiūrėti išvedimo informaciją.
		\end{tabular}										& 1, 3			\\ \hline

NF1 & 	Užtikrinti kalbos pakeitimą							& 1, 3			\\ \hline

NF2 & 	Prisijungimo vardo ilgis nemažesnis už 4				& 1, 3			\\ \hline 

NF3 & 	Slaptažodžio ilgis nemažesnis už 8					& 1, 3			\\ \hline 

NF4 & 
\begin{tabular}[c]{@{}l@{}}
Prisijungime pakeičiamai dalyvauja sistemoje registruotas\\
vartotojo slapyvardis arba el.paštas, bei slaptažodis.			
\end{tabular}												& 1, 3			\\ \hline 

NF5	&	Paštas bei prisjungimo vardas unikalūs				& 1, 3			\\ \hline

\end{tabular}
\end{frame}

\vspace{1cm}
\textbf{Moduliai} \\
1 - Vartotojų, 2 - P.O., 3 - Administratoriaus.


\vspace{1cm}
*Informacija - panelių kiekis, bendras jų plotas, likutis, pjovimo ilgis, detalių išdėstymas.

\section{Poreikių atsekamumo lentelė}
\begin{frame}
\centering
\hspace{-2.5cm}
\label{my-label}
\begin{tabular}{|l|c|}
\hline
\textbf{Užsakovo poreikiai}						& \textbf{Reikalavimai} 		\\ \hline

\begin{tabular}[c]{@{}l@{}}
Vartotojas nurodo reikalingų detalių ilgius (mm), 
aukščius (mm) ir kiekius.                                                                                                                                                                                                                                                                              
\end{tabular} 									& F9		                		\\ \hline

\begin{tabular}[c]{@{}l@{}}
Sistema optimizuoja detalių išpjovimą iš standartinių panelių 1200x2500.\\ 
Po to pakartoja šį procesą su standartiniais paneliais 1200x3050.\\ 
(Turėkite omenyje, kad pjovimo plotis yra 4 mm, todėl iš standartinės \\ 
panelės 1200x2500 negalima išpjauti 2 detalių 600x2500.)
\end{tabular}									& F10, F11					\\ \hline

\begin{tabular}[c]{@{}l@{}}
Sistema pateikia abiejų variantų informaciją vartotojui ir pasirenka, \\ 
kuris variantas jam labiau tinka. Vartotojui pasirinkus, sistema \\ 
turi parodyti ekrane pjovimo planą su galimybe jį atsispausdinti \\
(su atspausdintu planu vartotojas gali kreiptis į tiekėją, kad \\ 
išpjautų jam reikiamas detales).
\end{tabular}									& F12, F13	 				\\ \hline

Planuojama sistemą pateikti kaip paslaugą išoriniams vartotojams.                                                                                                                                                                                                                                                                                         												& F1, F2	, F3, F4, F5, F6	, F7	\\ \hline

Įvedami standartinių panelių dydžiai.			& F9							\\ \hline

\end{tabular}
\end{frame}

\clearpage





\end{document}
