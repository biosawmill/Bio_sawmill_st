\documentclass[a4paper,12pt]{article}
\usepackage[utf8x]{inputenc}
\usepackage[T1]{fontenc}

%\usepackage[T2A]{fontenc} % jei yra kirilica
\usepackage[hmargin={30mm,15mm},vmargin={20mm,20mm},bindingoffset=0mm]{geometry}
\usepackage[onehalfspacing]{setspace}
\usepackage[colorlinks=true, linkcolor=blue, citecolor=blue, urlcolor=blue, unicode]{hyperref}

%\parindent=7mm
\renewcommand{\refname}{Literatūros sąrašas} % article
%\renewcommand{\bibname}{Literatūros sąrašas} % report
\renewcommand{\contentsname}{Turinys}
\usepackage[T1]{fontenc} 

% Lukas paketai
\usepackage{booktabs}% http://ctan.org/pkg/booktabs
\newcommand{\tabitem}{~~\llap{\textbullet}~~}
\usepackage{graphicx}
\usepackage{indentfirst}
\usepackage{setspace}
\usepackage{placeins}
\usepackage{booktabs}% http://ctan.org/pkg/booktabs
\usepackage{tabularx}% http://ctan.org/pkg/tabularx
\usepackage[parfill]{parskip}
\usepackage[unicode]{hyperref}
\usepackage{hyperref}
\usepackage{tocloft}
\usepackage{graphicx}
\newcommand\AtPageUpperRight[1]{\AtPageUpperLeft{%
   \makebox[\paperwidth][r]{#1}}}
\usepackage[dotinlabels]{titletoc}
\usepackage[capposition=top]{floatrow}
\hypersetup{
    colorlinks,
    citecolor=black,
    filecolor=black,
    linkcolor=black,
    urlcolor=black
}
\usepackage{secdot}




\begin{document}
\graphicspath{ {/} }

\renewcommand{\cftdot}{.}	
\renewcommand{\cftsecleader}{\cftdotfill{\cftdotsep}}

\thispagestyle{empty} % nerasomas psl. nr


\begin{center}
 VILNIAUS UNIVERSITETAS 
 
MATEMATIKOS IR INFORMATIKOS FAKULTETAS

MATEMATINĖS INFORMATIKOS KATEDRA

\vspace{4cm}

Projekto vadovas \ \ \textbf{Lukas Tutkus} \\
\textbf{Julius Daukšas} \\
\textbf{Dominykas Smaliukas} \\
\textbf{Robert Stankevič} \\

\vspace{0.2cm}

Bioinformatikos studijų programos grupė BioSawmill



\vspace{3cm}
\textbf{\Large Reikalavimų specifikacija}\\

\vfill

Vilnius \ \  2015
\end{center}



\clearpage

\tableofcontents
\clearpage
\section{Funkciniai ir nefunkciniai reikalavimai}
\begin{frame}
\centering
\hspace{-2.5cm}
\label{my-label}
\begin{tabular}{|l|l|l|}
\hline
\textbf{ID}	& \textbf{Reikalavimai}						& \textbf{Moduliai}  \\ \hline

F1	& Registracija										& 1	     		\\ \hline

F2	& Prisijungimas prie sistemos						& 1, 3			\\ \hline

F3	& Atsijungimo nuo sistemos							& 1, 3			\\ \hline

F4	& Slaptažodžio atgavimas								& 1, 3			\\ \hline

F5	& Paskyros parametrų keitimas 	  					& 1, 3			\\ \hline 

F6	& Vartotojo paskyros deaktyvacija					& 1				\\ \hline

F7	& Vartotojo paskyros aktyvacija						& 1				\\ \hline

F8	& Vartotojo paskyros ištrinimas						& 3				\\ \hline

F9	& Pjovimo optimizacijos duomenų įvedimas				& 1, 3			\\ \hline

F10	& Pjūvio plano optimizavimas           	   			& 2				\\ \hline

F11	& Optimizuotų pjūvio planų informacijos pateikimas	& 2     			\\ \hline

F12	& Optimizuotų pjūvio planų variantų pasirinkimas		& 1, 3			\\ \hline

F13 & Pjūvio plano ruošinio atsisiuntimas				& 1, 3			\\ \hline

F14 	& Pjūvio plano informacijos išsaugojimas				& 1, 3			\\ \hline

F15	& Kalbos pasirinkimas(LT, EN)						& 1, 3			\\ \hline

NF1 & Užtikrinti kalbos pakeitimą						& 1, 3			\\ \hline

NF2 & Prisijungimo vardo ilgis nemažesnis už 4			& 1, 3			\\ \hline 

NF3 & Elektroninis paštas galioja						& 1, 3			\\ \hline 

NF3 & Slaptažodžio ilgis nemažesnis už 8					& 1, 3			\\ \hline 

NF4 & Prisijungimo duomenys išskyrus slaptažodį - unikalūs	& 1, 3			\\ \hline 

NF5 & 
\begin{tabular}[c]{@{}l@{}}
Prisijungime pakeičiamai dalyvauja sistemoje registruotas vartotojo slapyvardis\\
arba el.paštas, bei slaptažodis.			
\end{tabular}											& 1, 3			\\ \hline 


\end{tabular}
\end{frame}

\vspace{1cm}
\textbf{Moduliai} \\
1 - Vartotojų, 2 - P.O., 3 - Administratoriaus.


\vspace{1cm}
*Informacija - panelių kiekis, bendras jų plotas, likutis, pjovimo ilgis, detalių išdėstymas.

\section{Poreikių atsekamumo lentelė}
\begin{frame}
\centering
\hspace{-2.5cm}
\label{my-label}
\begin{tabular}{|l|c|}
\hline
\textbf{Užsakovo poreikiai}						& \textbf{Reikalavimai} 		\\ \hline

\begin{tabular}[c]{@{}l@{}}
Vartotojas nurodo reikalingų detalių ilgius (mm), 
aukščius (mm) ir kiekius.                                                                                                                                                                                                                                                                              
\end{tabular} 									& F9		                		\\ \hline

\begin{tabular}[c]{@{}l@{}}
Sistema optimizuoja detalių išpjovimą iš standartinių panelių 1200x2500.\\ 
Po to pakartoja šį procesą su standartiniais paneliais 1200x3050.\\ 
(Turėkite omenyje, kad pjovimo plotis yra 4 mm, todėl iš standartinės \\ 
panelės 1200x2500 negalima išpjauti 2 detalių 600x2500.)
\end{tabular}									& F10, F11					\\ \hline

\begin{tabular}[c]{@{}l@{}}
Sistema pateikia abiejų variantų informaciją vartotojui ir pasirenka, \\ 
kuris variantas jam labiau tinka. Vartotojui pasirinkus, sistema \\ 
turi parodyti ekrane pjovimo planą su galimybe jį atsispausdinti \\
(su atspausdintu planu vartotojas gali kreiptis į tiekėją, kad \\ 
išpjautų jam reikiamas detales).
\end{tabular}									& F12, F13	 				\\ \hline

Planuojama sistemą pateikti kaip paslaugą išoriniams vartotojams.                                                                                                                                                                                                                                                                                         												& F1, F2	, F3, F4, F5, F6	, F7	\\ \hline

Įvedami standartinių panelių dydžiai.			& F9							\\ \hline

\end{tabular}
\end{frame}

\clearpage





\end{document}
