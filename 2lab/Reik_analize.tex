\documentclass[a4paper,12pt]{article}
\usepackage[utf8x]{inputenc}
\usepackage[T1]{fontenc}
%\usepackage[T2A]{fontenc} % jei yra kirilica
\usepackage[hmargin={30mm,15mm},vmargin={20mm,20mm},bindingoffset=0mm]{geometry}
\usepackage[onehalfspacing]{setspace}
\usepackage[colorlinks=true, linkcolor=blue, citecolor=blue, urlcolor=blue, unicode]{hyperref}
%\parindent=7mm
\usepackage{graphicx}
\renewcommand{\refname}{Literatūros sąrašas} % article
%\renewcommand{\bibname}{Literatūros sąrašas} % report
\renewcommand{\contentsname}{Turinys}
\usepackage[T1]{fontenc} 

% Lukas paketai

\usepackage{eso-pic}
\usepackage{indentfirst}
\usepackage{setspace}
\usepackage{placeins}
\usepackage{booktabs}% http://ctan.org/pkg/booktabs
\usepackage{tabularx}% http://ctan.org/pkg/tabularx
\usepackage[parfill]{parskip}
\usepackage[unicode]{hyperref}
\usepackage{hyperref}
\usepackage{tocloft}
\usepackage{graphicx}
\newcommand\AtPageUpperRight[1]{\AtPageUpperLeft{%
   \makebox[\paperwidth][r]{#1}}}
\usepackage[dotinlabels]{titletoc}
\usepackage[capposition=top]{floatrow}
\hypersetup{
    colorlinks,
    citecolor=black,
    filecolor=black,
    linkcolor=black,
    urlcolor=black
}
\usepackage{secdot}




\begin{document}


\renewcommand{\cftdot}{.}	
\renewcommand{\cftsecleader}{\cftdotfill{\cftdotsep}}

\thispagestyle{empty} % nerasomas psl. nr


\begin{center}
 VILNIAUS UNIVERSITETAS 
 
MATEMATIKOS IR INFORMATIKOS FAKULTETAS

MATEMATINĖS INFORMATIKOS KATEDRA

\vspace{4cm}

Projekto vadovė \ \ \textbf{Aistė Čiplytė} \\
\textbf{Lukas Tutkus} \\
\textbf{Julius Daukšas} \\
\textbf{Dominykas Smaliukas} \\
\textbf{Robert Stankevič} \\

\vspace{0.2cm}

Bioinformatikos studijų programos grupė BioSawmill



\vspace{3cm}
\textbf{\Large Dvimatė pjovimo optimizacija}\\
\textbf{\Large Projekto planas}

\vfill

Vilnius \ \  2015
\end{center}



\clearpage

\tableofcontents
\clearpage
%\maketitle 
%\section*{Apžvalga}
%\addcontentsline{toc}{section}{Apžvalga} % rasoma turinyje
\section{Reikalavimai}

\begin{frame}
\centering

\label{my-label}
\begin{tabular}{|l|l|l|}
\hline
\textbf{ID}	& \textbf{Reikalavimai}						& \textbf{Moduliai}  \\ \hline

F1	& Registracijos forma								& 1     		\\ \hline

F2	& Prisijungimo forma									& 1			\\ \hline

F3	& Atsijungimo nuo sistemos mygtukas					& 1			\\ \hline

F4	& Vartotojo slaptažodžio pakeitimo forma				& 1			\\ \hline

F5	& Keisti paskyros duomenis	 	  					& 1			\\ \hline 

F6	& Pjovimo optimizacijos duomenų įvedimo laukai		& 1			\\ \hline

F7  & Papildomos standartinės panelės ivedimas 			& 1			\\ \hline

F8 & Pjūvio plano informacijos apdorojimas				& 2			\\ \hline

F9	& Pjūvio plano optimizavimas           	   			& 2			\\ \hline

F10	& Optimizuotų pjūvio planų informacijos pateikimas	& 2     		\\ \hline

F11	& Optimizuotų pjūvio planų variantų pasirinkimas		& 1			\\ \hline

F12	& Pasirinkto pjūvio planų ruošinių atsispausdinimas	& 1			\\ \hline

F13	& Pasirinkto pjūvio planų informacijos išsaugojimas	& 1			\\ \hline

F14	& Administratoriaus valdymo skydas	 	  			& 3			\\ \hline

NF1	& Kalbos pasirinkimas(LT, EN)						& 1			\\ \hline

\end{tabular}
\end{frame}

\vspace{1cm}
\textbf{Moduliai} \\
1-Varotojų, 2- P.O. algoritmo, 3 - Administratoriaus.


\vspace{1cm}
informacija - panelių kiekis, bendras jų plotas, likutis, detalių išdėstymas.

\section{Atsekamumo lentelė}
\begin{frame}
\centering
\hspace{0cm}
\label{my-label}
\begin{tabular}{|l|c|}
\hline
\textbf{Užsakovo poreikiai}						& \textbf{Reikalavimai} \\ \hline

\begin{tabular}[c]{@{}l@{}}
Vartotojas nurodo reikalingų detalių ilgius (mm), 
aukščius (mm) ir kiekius.                                                                                                                                                                                                                                                                              
\end{tabular} 									& F6                 	\\ \hline

\begin{tabular}[c]{@{}l@{}}
Sistema optimizuoja detalių išpjovimą iš standartinių panelių 1200x2500.\\ 
Po to pakartoja šį procesą su standartiniais paneliais 1200x3050.\\ 
(Turėkite omenyje, kad pjovimo plotis yra 4 mm, todėl iš standartinės \\ 
panelės 1200x2500 negalima išpjauti 2 detalių 600x2500.)
\end{tabular}									& F6, F7, F8, F9			\\ \hline

\begin{tabular}[c]{@{}l@{}}
Sistema pateikia abiejų variantų informaciją vartotojui ir pasirenka, \\ 
kuris variantas jam labiau tinka. Vartotojui pasirinkus, sistema \\ 
turi parodyti ekrane pjovimo planą su galimybe jį atsispausdinti \\
(su atspausdintu planu vartotojas gali kreiptis į tiekėją, kad \\ 
išpjautų jam reikiamas detales).
\end{tabular}									& F10, F11, F12	 		\\ \hline

Planuojama sistemą pateikti kaip paslaugą išoriniams vartotojams.                                                                                                                                                                                                                                                                                         												& F1, F2	, F3, F4, F5		\\ \hline

Įvedami standartinių panelių dydžiai.			& F6, F7						\\ \hline

\end{tabular}
\end{frame}


\section{Klasių diagramos}

\section{Naudojimo atvejų aprašai(Use-cases)}


% <<<<<<<<<<<<<<<<<<<<<<<<<<<<ŠABLONAS >>>>>>>>>>>>>>>>>>>>>>>>>>>>>>>>>>>>>>>>>>>>>>>>>>>

\begin{frame}
\centering
\hspace{-1cm}
\label{my-label}
\begin{tabular}{|l|l|}
\hline
\textbf{Pavadinimas}                                                                    &  \\ \hline
\textbf{Dalyviai}                                                                       &       \\ \hline
\textbf{Paskirtis}                                                                      &            \\ \hline
\textbf{Kritiškumas}                                                                    & \\ \hline
\textbf{Pradinės sąlygos}                                                               &        \\ \hline
\textbf{Rezultatas}                                                                     &           \\ \hline
\textbf{\begin{tabular}[c]{@{}l@{}}
Tipinė eiga ir kiti\\ galimi variantai
\end{tabular}} &   
\begin{tabular}[c]{@{}l@{}} 
	TEKSAS \end{tabular} \\ \hline
\end{tabular}
\end{frame}

% <<<<<<<<<<<<<<<<<<<<<<<<<<<<ŠABLONO PABAIGA >>>>>>>>>>>>>>>>>>>>>>>>>>>>>>>>>>>>>>>>>>>>

\begin{frame}
\centering
\hspace{-1cm}
\label{my-label}
\begin{tabular}{|l|l|}
\hline
\textbf{Pavadinimas}  		& Registracija \\ \hline
\textbf{Dalyviai}  			&  Vartotojas \\ \hline
\textbf{Paskirtis}  			&  Sukurti naujo vartotojo paskyrą\\ \hline
\textbf{Kritiškumas}			& Būtinas \\ \hline
\textbf{Pradinės sąlygos}   	& Nėra \\ \hline
\textbf{Rezultatas}   		& Sistemos duomenų bazėje sukurtas naujo vartotojo įrašą\\ \hline
\textbf{
	\begin{tabular}[c]{@{}l@{}}
		Tipinė eiga ir kiti\\ 
		galimi variantai
	\end{tabular}
} &   
\begin{tabular}[c]{@{}l@{}}
	1. Naujas vartotojas norintis naudotis pjovimo optimizacija turi \\
	prisiregistruoti, kur tai gali padaryti registravimosi lange užpildęs \\
	privalomus laukus: vartotojo vardą, slaptažodį, el.paštą.\\ 
	2. Paspaudžia registracijos mygtuką.\\
	3. Jeigu suvesti vartotojo vardas, bei paštas unikalus duomenų bazėje,\\
	tada sukuriamas vartotojo įrašas duomenų bazėje ir vartotojas\\
	nukreipiamas į jo naujos paskyros langą. \\
	Kitu atveju parodoma kas blogai vartotojo prašoma suvesti teisingus/leistinus duomenis ir\\ 
	bandyti iš naujo, t.y., pereinam į pirmą punktą. 
\end{tabular} \\ \hline
\end{tabular}
\end{frame}

\vspace{2cm}

\begin{frame}
\centering
\hspace{-1cm}
\label{my-label}
\begin{tabular}{|l|l|}
\hline
\textbf{Pavadinimas}  & Prisijungimas \\ \hline
\textbf{Dalyviai}  &  Vartotojas  \\ \hline
\textbf{Paskirtis}  &  Leisti prieeigą prie vartotojo paskyros \\ \hline
\textbf{Kritiškumas} & Būtinas\\ \hline
\textbf{Pradinės sąlygos}   & Vartotojas turi būti registruotas sistemos duomenų bazėje \\ \hline
\textbf{Rezultatas}   &  Vartotojas gavo prieeigą prie savo paskyros\\ \hline
\textbf{\begin{tabular}[c]{@{}l@{}}Tipinė eiga ir kiti\\ galimi variantai\end{tabular}} &   \begin{tabular}[c]{@{}l@{}}1. Prisijungimo lange vartotojas suveda prisijungimo duomenis:\\ vartotojo vardą ir slaptažodį.\\2. Paspaudžia prisijungimo mygtuką.\\ 3. Jeigu suvesti teisingi prisijungimo duomenys, vartotojas\\ nukreipiamas į jo paskyrs langą. Kitu  atveju vartotojui\\ pranešama, jog prisijungimo duomenys buvo klaidingi, taip pat \\ lange atsiranda mygtukas, turintis padėti vartotojui atgauti\\ savo prisijungimo duomenis. Vartotojas gali bandyti prisijungti\\ iš naujo arba spausti mygtuką prisijungimo duomenų atgavimui.\end{tabular} \\ \hline
\end{tabular}
\end{frame}

\begin{frame}
\centering
\hspace{-1cm}
\label{my-label}
\begin{tabular}{|l|l|}
\hline
\textbf{Pavadinimas}                                                                    & Atsijungimas nuo sistemos \\ \hline
\textbf{Dalyviai}                                                                       &   Vartotojas, administratorius    \\ \hline
\textbf{Paskirtis}                                                                      &  Nutraukti sistemos naudotojo sesiją        \\ \hline
\textbf{Kritiškumas}                                                                    &  Mažas\\ \hline
\textbf{Pradinės sąlygos}                                                               &  Sistemos naudotojas yra prisijungęs prie sistemos   \\ \hline
\textbf{Rezultatas}                                                                     &     Nutraukta sistemos naudotojo sesija      \\ \hline
\textbf{\begin{tabular}[c]{@{}l@{}}Tipinė eiga ir kiti\\ galimi variantai\end{tabular}} &     \begin{tabular}[c]{@{}l@{}}1. Sistemos naudotojas paspaudžia atsijungimo mygtuką,\\ kuris turi būti kiekviename svetainės tinklapio lange.\end{tabular} \\ \hline
\end{tabular}
\end{frame}

\begin{frame}
\centering
\hspace{-1cm}
\label{my-label}
\begin{tabular}{|l|l|}
\hline
\textbf{Pavadinimas}                                                                    &  Vartotojo slaptažodžio atgavimas\\ \hline
\textbf{Dalyviai}                                                                       & Vartotojas      \\ \hline
\textbf{Paskirtis}                                                                      &   Vartotojui sužinoti pamirštą slaptažodį   \\ \hline
\textbf{Kritiškumas}                                                                    & Būtinas \\ \hline
\textbf{Pradinės sąlygos}                                                               &  \begin{tabular}[c]{@{}l@{}} Vartotojos bando prisijungti, tačiau jam nepavyksta prisiminti\\ slaptažodžio. Jo prisijungimo lange atsiranda mygtukas pamiršto\\ slaptažodžio atgavimui \end{tabular} \\ \hline
\textbf{Rezultatas}                                                                     &   Vartotojas į el. paštą gauna savo slaptažodžio priminimą   \\ \hline
\textbf{\begin{tabular}[c]{@{}l@{}}Tipinė eiga ir kiti\\ galimi variantai\end{tabular}} &   \begin{tabular}[c]{@{}l@{}}1. Vartotojas paspaudžia pamiršto slaptažodžio atgavimo mygtuką.\\2. Atsidariusiame naujame lange vartotojos nurodo savo el. pašto\\ adresą, su kuriu yra užsiregistravęs sistemoje.\\ 3. Vartotojas spaudžia patvirtinimo mygtuką.\\ 4. Jei el. paštas egzistuoja sistemos duomenų bazėje, juo bus\\ nusiųstas slaptažodžio priminimas. Jei el. pašto atitikmens nerandama \\duomenų bazėje, vartotojas gali bandyti iš naujo įvesti el. pašto \\adresą arba naudoti kontaktinius duomenis, pateiktus kiekvieno\\ svetainės lango apačioje, prarastos paskyros prieigos atgavimui. \\\end{tabular} \\ \hline
\end{tabular}
\end{frame}

\begin{frame}
\centering
\hspace{-1cm}
\label{my-label}
\begin{tabular}{|l|l|}
\hline
\textbf{Pavadinimas}                                                                    &  Įvedami pjovimo optimizacijos parametrai\\ \hline
\textbf{Dalyviai}                                                                       &  Vartotojas  \\ \hline
\textbf{Paskirtis}                                                                      &  
\begin{tabular}[c]{@{}l@{}}
Sistemai gauti parametrus, pagal kuriuos turi būti optimizuotas pjovimo \\planas 
\end{tabular}      \\ \hline
\textbf{Kritiškumas}                                                                    & Būtinas\\ \hline
\textbf{Pradinės sąlygos}                                                               & Vartotojas yra prisijungęs ir atsidaręs pjovimo optimizavimui skirtą langą \\ \hline
\textbf{Rezultatas}                                                                     &  \begin{tabular}[c]{@{}l@{}}Įvesti parametrai reikalingi optimizacijai, galima pradėti vykdyti \\optimizavimo algoritmą\end{tabular} \\ \hline
\textbf{\begin{tabular}[c]{@{}l@{}}Tipinė eiga ir kiti\\ galimi variantai\end{tabular}} &   \begin{tabular}[c]{@{}l@{}} 1. Optimizavimo lange vartotojas nurodo reikalingų detalių ilgius (mm),\\ aukščius (mm) ir kiekius. Visi duomenys nurodomi sveikaisiais skaičiais. \\Vartotojas negalės pereiti prie sekančio žingsnio, jeigu bent vienas \\parametras bus įvestas neteisingai. \\2. Vartotojas nurodo, pagal kokius standartinius paneliu dydžius \\ turėtų būti optimizuojamas pjovimo planas. Jeigu norimų standartinių \\ panelių dydžių nenurodo, optimizacija atliekama pagal 1200x2500 ir \\
1200x3050 standartiniu panelių dydžius. Vartotojas gali gauti iki triejų\\ optmizavimo planų (pagal skirtingus panelių dydžius).\end{tabular} \\ \hline
\end{tabular}
\end{frame}

\begin{frame}
\centering
\hspace{-1cm}
\label{my-label}
\begin{tabular}{|l|l|}
\hline
\textbf{Pavadinimas}                                                                    &  Įvesti papildomus standartinių panelių dydžius\\ \hline
\textbf{Dalyviai}                                                                       &  Vartotojas  \\ \hline
\textbf{Paskirtis}                                                                      &  
\begin{tabular}[c]{@{}l@{}}
Vartotojui nurodyti panelės, su kuriai jis nori rasti optimalų pjovimo \\planą, dydį.
\end{tabular}      \\ \hline
\textbf{Kritiškumas}                                                                    & Būtinas\\ \hline
\textbf{Pradinės sąlygos}                                                               &  \begin{tabular}[c]{@{}l@{}}Vartotojas yra prisijungęs ir atsidaręs pjovimo optimizavimui skirtą langą.\\ Suvesti detalių matmenys ir norimas jų kiekis.\end{tabular}  \\ \hline
\textbf{Rezultatas}                                                                     &  \begin{tabular}[c]{@{}l@{}}Vartotojas turi galimybę nurodyti panelių, iš kurių nori pjauti detales,\\ dydį\end{tabular} \\ \hline
\textbf{\begin{tabular}[c]{@{}l@{}}Tipinė eiga ir kiti\\ galimi variantai\end{tabular}} &   \begin{tabular}[c]{@{}l@{}} 1. Vartotojas peržiuri siūlomų panelių dydžių sąrašą. Optimizacijai \\gali pasirinkti vieną iš šių siūlomų variantų arba nurodyti kitokius \\matmenis. Antruoju atveju būtų taikomi įvedamų matmenų apribojimai,\\pavyzdžiui, matmuo negali būti didesnis negu 5000mm ar mažesnis\\ negu 2000mm. \end{tabular} \\ \hline
\end{tabular}
\end{frame}

\begin{frame}
\centering
\hspace{-1cm}
\label{my-label}
\begin{tabular}{|l|l|}
\hline
\textbf{Pavadinimas}                                                                    &  Optimizuotų pjūvio planų pasirinkimas\\ \hline
\textbf{Dalyviai}                                                                       &   Vartotojas    \\ \hline
\textbf{Paskirtis}                                                                      &  
\begin{tabular}[c]{@{}l@{}}
Leisti vartotojui pasirinkti, kuris pjūvio optimizacijos planas \\ jam labiau priimtinas 
\end{tabular}  \\ \hline
\textbf{Kritiškumas}                                                                    & Būtinas\\ \hline
\textbf{Pradinės sąlygos}                                                               & 
\begin{tabular}[c]{@{}l@{}}
Vartotojas yra prisijungęs. Sistema baigusi optimizuoti pjūvio \\planus/planą pagal vartotojo nurodytus optimizacijos parametrus \\optimizacijos lange. Priklausomai nuo optimizacijos parametrų \\vartotojas gauna nuo vieno iki triejų skirtingų pjovimo\\ optmizacijos planų
\end{tabular} \\ \hline
\textbf{Rezultatas}                                                                     &   \begin{tabular}[c]{@{}l@{}} Vartotojas savo naršyklės lange mato vieną, detalų pjovimo\\optimizacijos planą ir turi galimybę jį atsispausdinti\end{tabular}     \\ \hline
\textbf{\begin{tabular}[c]{@{}l@{}}Tipinė eiga ir kiti\\ galimi variantai\end{tabular}} &   \begin{tabular}[c]{@{}l@{}} 1. Vartotojas pasirenka vieną iš jam pateiktų pjovimo planų\\ ir patvirtina, jog nori matyti konkretų planą.\\2. Vartotojo naršyklės lange atsiranda detalus, pasirinktasis\\ pjovimo planas ir mygtukas, leidžiantis vartotojui parsisiųsti,\\ spausdinimui skirtą plano kopiją. Pastarasis planas \\ bus saugomas sistemos duomenų bazėje visma laikui, kai, tuo \\tarpu, nepasirinktieji optimizuoti pjovimo planai yra saugomi\\ sistemos duomenų bazėje iki vartotojo sesijos pabaigos.\end{tabular} \\ \hline
\end{tabular}
\end{frame}

\begin{frame}
\centering
\hspace{-1cm}
\label{my-label}
\begin{tabular}{|l|l|}
\hline
\textbf{Pavadinimas}                                                                    &  Pasirinkto pjūvio plano ruošinio gavimas\\ \hline
\textbf{Dalyviai}                                                                       &   Vartotojas    \\ \hline
\textbf{Paskirtis}                                                                      &  
\begin{tabular}[c]{@{}l@{}}
Leisti vartotojui gauti spausdinimui skirtą pjūvio plano kopiją
\end{tabular}  \\ \hline
\textbf{Kritiškumas}                                                                    & Būtinas\\ \hline
\textbf{Pradinės sąlygos}                                                               & 
\begin{tabular}[c]{@{}l@{}}
Vartotojas yra prisijungęs. Sistema baigusi optimizuoti pjūvio \\planus/planą pagal vartotojo nurodytus optimizacijos parametrus \\optimizacijos lange. Vartotojas yra pasirinkęs pjūvio optimizavimo\\ planą, kurį nori atsispausdinti
\end{tabular} \\ \hline
\textbf{Rezultatas}                                                                     &   \begin{tabular}[c]{@{}l@{}} Vartotojas savo prietaise turės spausdinimui paruoštą\\ pjūvio planą\end{tabular}     \\ \hline
\textbf{\begin{tabular}[c]{@{}l@{}}Tipinė eiga ir kiti\\ galimi variantai\end{tabular}} &   \begin{tabular}[c]{@{}l@{}} 1. Vartotojas paspaudžia mygtuką, skirtą gauti spausdinimui \\paruoštą planą.\\2. Į vartotojo prietaisą atsiunčiamas pjuvio planas.\end{tabular} \\ \hline
\end{tabular}
\end{frame}

\begin{frame}
\centering
\hspace{-1cm}
\label{my-label}
\begin{tabular}{|l|l|}
\hline
\textbf{Pavadinimas}                                                                    &  Keisti paskyros parametrus\\ \hline
\textbf{Dalyviai}                                                                       &     Vartotojas  \\ \hline
\textbf{Paskirtis}                                                                      &      Leisti vartotojui keisti savo paskyruos duomenis      \\ \hline
\textbf{Kritiškumas}                                                                    & Vidutinis\\ \hline
\textbf{Pradinės sąlygos}                                                               &     \begin{tabular}[c]{@{}l@{}} Vartotojas yra prisijungęs. Vartotojui prisijungus, jis\\ yra nukreipiamas į savo paskyros langą\end{tabular}  \\ \hline
\textbf{Rezultatas}                                                                     &   Pasikeitę vartotojo paskyros parametrai        \\ \hline
\textbf{\begin{tabular}[c]{@{}l@{}}Tipinė eiga ir kiti\\ galimi variantai\end{tabular}} &   \begin{tabular}[c]{@{}l@{}} 1. Vartotojas spaudžia paskyros redagavimo mygtuką.\\ 2. Vartotojas pakeičia norimus paskyros duomenis.\\ 3. Vartotojas spaudžia pakeitimų išsaugojimo mygtuką.\\4. Jei pakeitimai buvo leistini, pavyzdžiui, vartotojo\\ vardas nebuvo pakeistas į tuščią reikšmę ir pan., tada \\pakeitimai išsaugomi. Kitu atveju pažymimi\\ neleistinas reikšmes įgiję laukai. Vartotojas jas turi \\pakeisti, kad galėtų išsaugoti pakitimus.\end{tabular} \\ \hline
\end{tabular}
\end{frame}

\section{Sistemos elgsenos aprašymas}



\end{document}
