\documentclass[a4paper,12pt]{article}
\usepackage[utf8x]{inputenc}
\usepackage[T1]{fontenc}
%\usepackage[T2A]{fontenc} % jei yra kirilica
\usepackage[hmargin={30mm,15mm},vmargin={20mm,20mm},bindingoffset=0mm]{geometry}
\usepackage[onehalfspacing]{setspace}
\usepackage[colorlinks=true, linkcolor=blue, citecolor=blue, urlcolor=blue, unicode]{hyperref}
%\parindent=7mm
\usepackage{graphicx}
\renewcommand{\refname}{Literatūros sąrašas} % article
%\renewcommand{\bibname}{Literatūros sąrašas} % report
\renewcommand{\contentsname}{Turinys}
\usepackage[T1]{fontenc} 

% Lukas paketai

\usepackage{eso-pic}
\usepackage{indentfirst}
\usepackage{setspace}
\usepackage{placeins}
\usepackage{booktabs}% http://ctan.org/pkg/booktabs
\usepackage{tabularx}% http://ctan.org/pkg/tabularx
\usepackage[parfill]{parskip}
\usepackage[unicode]{hyperref}
\usepackage{hyperref}
\usepackage{tocloft}
\usepackage{graphicx}
\newcommand\AtPageUpperRight[1]{\AtPageUpperLeft{%
   \makebox[\paperwidth][r]{#1}}}
\usepackage[dotinlabels]{titletoc}
\usepackage[capposition=top]{floatrow}
\hypersetup{
    colorlinks,
    citecolor=black,
    filecolor=black,
    linkcolor=black,
    urlcolor=black
}
\usepackage{secdot}




\begin{document}


\renewcommand{\cftdot}{.}	
\renewcommand{\cftsecleader}{\cftdotfill{\cftdotsep}}

\thispagestyle{empty} % nerasomas psl. nr


\begin{center}
 VILNIAUS UNIVERSITETAS 
 
MATEMATIKOS IR INFORMATIKOS FAKULTETAS

MATEMATINĖS INFORMATIKOS KATEDRA

\vspace{4cm}

Projekto vadovė \ \ \textbf{Aistė Čiplytė} \\
\textbf{Lukas Tutkus} \\
\textbf{Julius Daukšas} \\
\textbf{Dominykas Smaliukas} \\
\textbf{Robert Stankevič} \\

\vspace{0.2cm}

Bioinformatikos studijų programos grupė BioSawmill



\vspace{3cm}
\textbf{\Large Dvimatė pjovimo optimizacija}\\
\textbf{\Large Projekto planas}

\vfill

Vilnius \ \  2015
\end{center}



\clearpage

\tableofcontents
\clearpage
%\maketitle 
%\section*{Apžvalga}
%\addcontentsline{toc}{section}{Apžvalga} % rasoma turinyje
\section{Reikalavimai}

\begin{frame}
\centering

\label{my-label}
\begin{tabular}{|l|l|l|}
\hline
\textbf{ID}              & \textbf{Reikalavimai}               & \textbf{Moduliai}  \\ \hline
F1                       & Registracijos forma                	   & 1      \\ \hline

F2                       & Prisijungimo forma                		   & 1      \\ \hline

F3						& Atsijungimo nuo sistemos mygtukas 	       & 1		\\ \hline

F4                       & Vartotojo slaptažodžio pakeitimo forma   & 1      \\ \hline

F5						& Prisijungimo duomenų išsaugojimas duomenų bazėje 
																   & 2		\\ \hline


F6                       & Detalių duomenų įvedimas           	   & 1            \\ \hline

F7                       & Optimizuotų pjuvio planų informacijos pateikimas 																  						 & 2      \\ \hline

F8                       & Optimizuotų pjuvio planų variantų pasirinkimas 																   					   & 1      \\ \hline

F9                      & Pjūvio plano informacijos išsaugojimas 
															      & 2       \\ \hline

F10                      & Pasirinkto pjūvio plano ruošinio gavimas& 1       \\ \hline


F11						& Administratoriaus valdymo skydas	 	  & 3			\\ \hline


NF1                      & Kalbos pasirinkimas(LT, EN)       	     & 1, 2, 3                 \\ \hline
\end{tabular}
\end{frame}

\vspace{1cm}
\textbf{Moduliai} \\
1-Varotojų, 2- Sistemos, 3 - Administratoriaus.


\vspace{1cm}
informacija - panelių kiekis, bendras jų plotas, likutis, detalių išdėstymas.

\section{Atsekamumo lentelė}
\begin{frame}
\centering
\hspace{0cm}
\label{my-label}
\begin{tabular}{|l|c|}
\hline
\multicolumn{1}{|c|}{\textbf{Užsakovo poreikiai}}                                                                                                                                                                                                                                                                                                         & \textbf{Reikalavimai} \\ \hline
Vartotojas nurodo reikalingų detalių ilgius (mm), aukščius (mm) ir kiekius.                                                                                                                                                                                                                                                                               &                       \\ \hline
\begin{tabular}[c]{@{}l@{}}Sistema optimizuoja detalių išpjovimą iš standartinių panelių 1200x2500.\\ Po to pakartoja šį procesą su standartiniais paneliais 1200x3050.\\ (Turėkite omenyje, kad pjovimo plotis yra 4 mm, todėl iš standartinės \\ panelės1200x2500 negalima išpjauti 2 detalių 600x2500.)\end{tabular}                                   &                       \\ \hline
\begin{tabular}[c]{@{}l@{}}Sistema pateikia abiejų variantų informaciją vartotojui ir pasirenka, \\ kuris variantas jam labiau tinka. Vartotojui pasirinkus, sistema \\ turi parodyti ekranepjovimo planą su galimybe jį atsispausdinti \\ (su atspausdintu planu vartotojasgali kreiptis į tiekėją, kad \\ išpjautų jam reikiamas detales).\end{tabular} &                       \\ \hline
Planuojama sistemą pateikti kaip paslaugą išoriniams vartotojams.                                                                                                                                                                                                                                                                                         &                       \\ \hline
Įvedami standartinių panelių dydžiai.                                                                                                                                                                                                                                                                                                                     &                       \\ \hline
\end{tabular}
\end{frame}


\section{Klasių diagramos}

\section{Naudojimo atvejų aprašai(Use-cases)}


% <<<<<<<<<<<<<<<<<<<<<<<<<<<<ŠABLONAS >>>>>>>>>>>>>>>>>>>>>>>>>>>>>>>>>>>>>>>>>>>>>>>>>>>

\begin{frame}
\centering
\hspace{-1cm}
\label{my-label}
\begin{tabular}{|l|l|}
\hline
\textbf{Pavadinimas}                                                                    &  \\ \hline
\textbf{Dalyviai}                                                                       &       \\ \hline
\textbf{Paskirtis}                                                                      &            \\ \hline
\textbf{Kritiškumas}                                                                    & \\ \hline
\textbf{Pradinės sąlygos}                                                               &        \\ \hline
\textbf{Rezultatas}                                                                     &           \\ \hline
\textbf{\begin{tabular}[c]{@{}l@{}}Tipinė eiga ir kiti\\ galimi variantai\end{tabular}} &              \\ \hline
\end{tabular}
\end{frame}

% <<<<<<<<<<<<<<<<<<<<<<<<<<<<ŠABLONO PABAIGA >>>>>>>>>>>>>>>>>>>>>>>>>>>>>>>>>>>>>>>>>>>>

\begin{frame}
\centering
\hspace{-1cm}
\label{my-label}
\begin{tabular}{|l|l|}
\hline
\textbf{Pavadinimas}  & Registracija \\ \hline
\textbf{Dalyviai}  &  Vartotojas \\ \hline
\textbf{Paskirtis}  &  Sukurti naujo vartotojo paskyrą\\ \hline
\textbf{Kritiškumas} & Būtinas \\ \hline
\textbf{Pradinės sąlygos}   & Nėra \\ \hline
\textbf{Rezultatas}   & Sistemos duomenų bazėje sukurtas naujo vartotojo įrašą\\ \hline
\textbf{\begin{tabular}[c]{@{}l@{}}Tipinė eiga ir kiti\\ galimi variantai\end{tabular}} &   \begin{tabular}[c]{@{}l@{}}1. Būsimas vartotojas registracijos puslapyje užpildo \\ privalomus laukus: vartotojo vardą, slaptažodį, el.paštą.\\ 2. Paspaudžia registracijos mygtuką.\\3. Jeigu suvesti duomenys buvo leistini, sukuriamas naujo\\ vartotojo įrašas duomenų bazėje ir vartotojas\\nukreipiamas į jo naujos paskyros langą. Kitu atveju\\ vartotojo prašoma suvesti teisingus/leistinus duomenis ir\\ bandyti iš naujo, t.y., pereinam į 1 punktą. \end{tabular} \\ \hline
\end{tabular}
\end{frame}



\begin{frame}
\centering
\hspace{-1cm}
\label{my-label}
\begin{tabular}{|l|l|}
\hline
\textbf{Pavadinimas}  & Prisijungimas \\ \hline
\textbf{Dalyviai}  &  Vartotojas  \\ \hline
\textbf{Paskirtis}  &  Leisti prieeigą prie vartotojo paskyros \\ \hline
\textbf{Kritiškumas} & Būtinas\\ \hline
\textbf{Pradinės sąlygos}   & Vartotojas turi būti registruotas sistemos duomenų bazėje \\ \hline
\textbf{Rezultatas}   &  Vartotojas gavo prieeigą prie savo paskyros\\ \hline
\textbf{\begin{tabular}[c]{@{}l@{}}Tipinė eiga ir kiti\\ galimi variantai\end{tabular}} &   \begin{tabular}[c]{@{}l@{}}1. Prisijungimo lange vartotojas suveda prisijungimo duomenis:\\ vartotojo vardą ir slaptažodį.\\2. Paspaudžia prisijungimo mygtuką.\\ 3. Jeigu suvesti teisingi prisijungimo duomenys, vartotojas\\ nukreipiamas į jo paskyrs langą. Kitu  atveju vartotojui\\ pranešama, jog prisijungimo duomenys buvo klaidingi, taip pat \\ lange atsiranda mygtukas, turintis padėti vartotojui atgauti\\ savo prisijungimo duomenis. Vartotojas gali bandyti prisijungti\\ iš naujo arba spausti mygtuką prisijungimo duomenų atgavimui.\end{tabular} \\ \hline
\end{tabular}
\end{frame}

\section{Sistemos elgsenos aprašymas}



\end{document}
