\documentclass[a4paper,12pt]{article}
\usepackage[utf8x]{inputenc}
\usepackage[T1]{fontenc}
%\usepackage[T2A]{fontenc} % jei yra kirilica
\usepackage[hmargin={30mm,15mm},vmargin={20mm,20mm},bindingoffset=0mm]{geometry}
\usepackage[onehalfspacing]{setspace}
\usepackage[colorlinks=true, linkcolor=blue, citecolor=blue, urlcolor=blue, unicode]{hyperref}
%\parindent=7mm
\usepackage{graphicx}
\renewcommand{\refname}{Literatūros sąrašas} % article
%\renewcommand{\bibname}{Literatūros sąrašas} % report
\renewcommand{\contentsname}{Turinys}
\usepackage[T1]{fontenc} 

% Lukas paketai

\usepackage{eso-pic}
\usepackage{indentfirst}
\usepackage{setspace}
\usepackage{placeins}
\usepackage{booktabs}% http://ctan.org/pkg/booktabs
\usepackage{tabularx}% http://ctan.org/pkg/tabularx
\usepackage[parfill]{parskip}
\usepackage[unicode]{hyperref}
\usepackage{hyperref}
\usepackage{tocloft}
\usepackage{graphicx}
\newcommand\AtPageUpperRight[1]{\AtPageUpperLeft{%
   \makebox[\paperwidth][r]{#1}}}
\usepackage[dotinlabels]{titletoc}
\usepackage[capposition=top]{floatrow}
\hypersetup{
    colorlinks,
    citecolor=black,
    filecolor=black,
    linkcolor=black,
    urlcolor=black
}
\usepackage{secdot}




\begin{document}


\renewcommand{\cftdot}{.}	
\renewcommand{\cftsecleader}{\cftdotfill{\cftdotsep}}

\thispagestyle{empty} % nerasomas psl. nr


\begin{center}
 VILNIAUS UNIVERSITETAS 
 
MATEMATIKOS IR INFORMATIKOS FAKULTETAS

MATEMATINĖS INFORMATIKOS KATEDRA

\vspace{4cm}

Projekto vadovė \ \ \textbf{Aistė Čiplytė} \\
\textbf{Lukas Tutkus} \\
\textbf{Julius Daukšas} \\
\textbf{Dominykas Smaliukas} \\
\textbf{Robert Stankevič} \\

\vspace{0.2cm}

Bioinformatikos studijų programos grupė BioSawmill



\vspace{3cm}
\textbf{\Large Dvimatė pjovimo optimizacija}\\
\textbf{\Large Projekto planas}

\vfill

Vilnius \ \  2015
\end{center}



\clearpage

\tableofcontents
\clearpage
%\maketitle 
%\section*{Apžvalga}
%\addcontentsline{toc}{section}{Apžvalga} % rasoma turinyje
\section{Reikalavimai}

\begin{frame}
\centering

\label{my-label}
\begin{tabular}{|l|l|l|}
\hline
\textbf{ID}              & \textbf{Reikalavimai}               & \textbf{Langas} \\ \hline
F1                       & Registracija                        & 1               \\ \hline
F2                       & Prisijungimas                       & 2               \\ \hline
F3                       & Vartotojo slaptažodžio pakeitimas   & 2               \\ \hline
F4                       & Kalbos pasirinkimas(LT, EN)         & 1, 2, 3, 4      \\ \hline
F5                       & Detalių duomenų įvedimas            & 3               \\ \hline
F6                       & Optimizacijos išvedimas             & 3               \\ \hline
F7                       & Pagalbą vartotojui                  & 4               \\ \hline
\multicolumn{1}{|c|}{F8} & Laiškų išsiuntimas                  & 4               \\ \hline
F9                       & Ruošinių generavimas                & 3               \\ \hline
F10                      & Optimizacijos ruošinių išsaugojimas & 3               \\ \hline
F11                      & Rezultatų gavimas vartotojui        & 3               \\ \hline
F12                      & Generavimas PDF išsaugojimas        & 3               \\ \hline
NF1                      & Serverio palaikymas                 &                 \\ \hline
NF2                      & Modernus, kokybiškas dizainas       &                 \\ \hline
\end{tabular}
\end{frame}


\section{Atsekamumo lentelė}
\begin{frame}
\centering
\hspace{0cm}
\label{my-label}
\begin{tabular}{|l|c|}
\hline
\multicolumn{1}{|c|}{\textbf{Užsakovo poreikiai}}                                                                                                                                                                                                                                                                                                         & \textbf{Reikalavimai} \\ \hline
Vartotojas nurodo reikalingų detalių ilgius (mm), aukščius (mm) ir kiekius.                                                                                                                                                                                                                                                                               &                       \\ \hline
\begin{tabular}[c]{@{}l@{}}Sistema optimizuoja detalių išpjovimą iš standartinių panelių 1200x2500.\\ Po to pakartoja šį procesą su standartiniais paneliais 1200x3050.\\ (Turėkite omenyje, kad pjovimo plotis yra 4 mm, todėl iš standartinės \\ panelės1200x2500 negalima išpjauti 2 detalių 600x2500.)\end{tabular}                                   &                       \\ \hline
\begin{tabular}[c]{@{}l@{}}Sistema pateikia abiejų variantų informaciją vartotojui ir pasirenka, \\ kuris variantas jam labiau tinka. Vartotojui pasirinkus, sistema \\ turi parodyti ekranepjovimo planą su galimybe jį atsispausdinti \\ (su atspausdintu planu vartotojasgali kreiptis į tiekėją, kad \\ išpjautų jam reikiamas detales).\end{tabular} &                       \\ \hline
Planuojama sistemą pateikti kaip paslaugą išoriniams vartotojams.                                                                                                                                                                                                                                                                                         &                       \\ \hline
Įvedami standartinių panelių dydžiai.                                                                                                                                                                                                                                                                                                                     &                       \\ \hline
\end{tabular}
\end{frame}


\section{Klasių diagramos}

\section{Naudojimo atvejų aprašai(Use-cases)}


% <<<<<<<<<<<<<<<<<<<<<<<<<<<<ŠABLONAS >>>>>>>>>>>>>>>>>>>>>>>>>>>>>>>>>>>>>>>>>>>>>>>>>>>

\begin{frame}
\centering
\hspace{-1cm}
\label{my-label}
\begin{tabular}{|l|l|}
\hline
\textbf{Pavadinimas}                                                                    &  \\ \hline
\textbf{Dalyviai}                                                                       &       \\ \hline
\textbf{Paskirtis}                                                                      &            \\ \hline
\textbf{Kritiškumas}                                                                    & \\ \hline
\textbf{Pradinės sąlygos}                                                               &        \\ \hline
\textbf{Rezultatas}                                                                     &           \\ \hline
\textbf{\begin{tabular}[c]{@{}l@{}}Tipinė eiga ir kiti\\ galimi variantai\end{tabular}} &              \\ \hline
\end{tabular}
\end{frame}

% <<<<<<<<<<<<<<<<<<<<<<<<<<<<ŠABLONO PABAIGA >>>>>>>>>>>>>>>>>>>>>>>>>>>>>>>>>>>>>>>>>>>>

\section{Sistemos elgsenos aprašymas}



\end{document}
